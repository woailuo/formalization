\section{Proof of Lemmas}
\label{sec:proof}

\begin{lemma}
\label{lem:okFpreserved}
If \( \langle P, F \rangle \xlongrightarrow{\rho} \langle P', F' \rangle\) and \( OK(F)\), then \(OK(F')\)
\end{lemma}

\begin{proof}
By induction on \( \langle P, F \rangle \xlongrightarrow{\rho} \langle P', F' \rangle\).
  \begin{itemize}

  \item Case \( P = {\bf 0};P'\) and \( \langle {\bf 0};P', F \rangle \rightarrow \langle P', F \rangle \)
    
      We need to prove \(OK(F')\). From assumption, we have that
      \(OK(F)\) holds and F' is the same as F. Therefore, \(OK(F')\)
      holds.

    \item Case \( P = \Malloc\) and \( \langle \Malloc, F \rangle \xlongrightarrow{\Malloc} \langle {\bf 0}, F \rangle \) 

      Similiar to above.

    \item Case \( P = \Free\) and \( \langle \Free, F \rangle \xlongrightarrow{\Free} \langle {\bf 0}, F \rangle \)

      Similiar to above.

    \item Case \( P = \Sirx(P_1,P_2)\) and \( \frac{\snnull \notin F
      \andalso \scon\Sirx \in F} { \langle \Sirx(P_1, P_2), F \rangle
      \rightarrow \langle P_1, F\cup \{\snull \} \rangle } \)
      
      We need to prove \( OK(F\cup\{ \snull\})\). From the assumption,
      we have that \(OK(F)\) holds and \( \snnull \notin F\). Then, we
      have \(F\cup\{ \snull\}\) does not contain both \(\snull\) and
      \( \snnull\). Therefore, \( OK(F\cup\{ \snull\})\) holds.

    \item Case \( P = \Sirx(P_1,P_2)\) and \( \frac{\scon\Sirx \notin F}
      { \langle \Sirx(P_1, P_2), F \rangle
      \rightarrow \langle P_1, F \rangle } \) \\
      We need to prove \( OK(F)\). From the assumption, \(OK(F)\) holds.

    \item Case \( P = \Sirx(P_1,P_2)\) and \( \frac{\scon\Sirx \notin F}
      { \langle \Sirx(P_1, P_2), F \rangle
      \rightarrow \langle P_2, F \rangle } \) \\
      We need to prove \( OK(F)\). From the assumption, \(OK(F)\) holds.

    \item Case \( P = \Sirx(P_1,P_2)\) and \( \frac{ \snull \in F \andalso \scon\Sirx \in F}
      { \langle \Sirx(P_1, P_2), F \rangle
      \rightarrow \langle P_1, F \rangle } \) \\
      We need to prove \( OK(F)\). From the assumption, \(OK(F)\) holds.

    \item Case \( P = \scon\Sirx P'\) and \(  \langle \scon\Sirx
      P', F \rangle \rightarrow \langle P';\Endconst, F\cup\{\scon\Sirx\} \rangle \) \\
      We need to prove \( OK(F\cup\{\scon\Sirx\})\). From the
      assumption, we have \(OK(F)\) holds. Therefore,
      \(F\cup\{\scon\Sirx\}\) does not contain both \(\snull\) and \(
      \snnull\). \( OK(F\cup\{ \scon\Sirx\})\) holds.

    \item Case \( P = \Sirx(P_1,P_2)\) and \( \frac{ \snull \notin F \andalso \scon\Sirx \notin F}
      { \langle \Sirx(P_1, P_2), F \rangle
      \rightarrow \langle P_2, F\cup\{\snnull\} \rangle } \) \\
      We need to prove \( OK(F\cup\{\snnull\})\). From the assumption,
      we have that \(OK(F)\) holds and \( \snull \not in F\). Then we
      get \(F\cup\{\snnull\}\) does not contain both \(\snull\) and
      \(\snnull\). Therefore, \( OK(F\cup\{\snnull\})\) holds.


    \item Case \( P = \mu\alpha.P'\) and \( \langle \mu\alpha.P', F \rangle
      \rightarrow \langle [\mu\alpha.P']P', F \rangle \) \\
      We need to prove \( OK(F)\). From the assumption, we have that
      \(OK(F)\) holds.

    \item Case \( P = P_1 + P_2\) and \( \langle P_1 + P_2, F \rangle
      \rightarrow \langle P_1, F \rangle \) \\
      We need to prove \( OK(F)\). From the assumption, we have that
      \(OK(F)\) holds.

    \item Case \( P = P_1 + P_2\) and \(  \langle P_1 + P_2, F \rangle
      \rightarrow \langle P_2, F \rangle \) \\
      We need to prove \( OK(F)\). From the assumption, we have that
      \(OK(F)\) holds.

    \item Case \( P = P_1;P_2\) and \( \frac{ \langle P_1, F \rangle \xlongrightarrow{\rho} \langle P_1', F' \rangle }
      {\langle P_1;P_2, F \rangle \xlongrightarrow{\rho} \langle P_1';P_2, F' \rangle} \) \\
    We need to prove \(OK(F')\). By IH, we have \( \langle P_1, F
    \rangle \xlongrightarrow{\rho} \langle P_1', F' \rangle \) and \(
    OK(F) \) holds, then \(OK(F')\) holds.
      
      
  \end{itemize}
\end{proof}



\begin{lemma}
\label{lem:okPreserved}
If \(\OK_n(P, F)\) and \( \langle P, F \rangle \xlongrightarrow{\rho} \langle P', F' \rangle\), then
\begin{itemize}
\item \(\OK_{n-1}(P', F')\) if \(\rho = \Malloc\),
\item \(\OK_{n+1}(P', F')\) if \(\rho = \Free\),
\item \(\OK_n(P')\) if \(\rho = \mbox{Otherwise}  \)
\end{itemize}
\end{lemma}

\begin{proof}
By induction on \(\langle P, F \rangle \xlongrightarrow{\rho} \langle P', F' \rangle \).

\begin{itemize}

\item Case \(P = {\bf 0};P'\) and \( \langle P', F \rangle \rightarrow \langle P', F \rangle\)

  We need to prove \(\OK_n(P', F)\).  Assume that \(\OK_n(P', F)\)
  does not hold. Then, we have (1) \( \exists \sigma \) and \(Q\) s.t. \(
  \langle P', F \rangle \xlongrightarrow{\sigma} \langle Q, F' \rangle
  \) and \(\sharp_{m}(\sigma) - \sharp_{f}(\sigma) > n\) or (2) \(
  OK(F)\) does not hold.

  From the definition of that \(OK({\bf 0};P', F)\) holds, we have (1)
  \( \langle {\bf 0};P', F \rangle \rightarrow \langle P', F \rangle
  \xlongrightarrow{\sigma} \langle Q, F' \rangle \) s.t.
  \(\sharp_m(\sigma) - \sharp_f(\sigma) \le n \) and (2) \(OK(F)\),
  which are in contradiction to the assumption. Therefore, \(\OK_n(P',
  F)\) holds.

\item Case \(P = \Malloc\) and \( \langle \Malloc, F \rangle \xlongrightarrow{\Malloc} \langle {\bf 0}, F \rangle\)

  We need to prove \(\OK_{n-1}({\bf 0}, F)\), which means we need to
  prove (1) \( \forall \sigma \) and \(Q\) s.t. \( \langle {\bf 0}, F
  \rangle \xlongrightarrow{\sigma} \langle Q, F' \rangle \) and
  \(\sharp_{m}(\sigma) - \sharp_{f}(\sigma) \le n\) and (2) \( OK(F)\)
  holds.  There is no \(Q\) and \(\sigma\) s.t. \(\langle {\bf 0}, F
  \rangle \xlongrightarrow{\sigma} \langle Q, F \rangle \), so (1)
  holds.  \(OK(F)\) holds from that \(OK(\Malloc, F)\)
  holds. Therefore, \(OK({\bf 0}, F)\) holds.
  
\item Case \(P = \Free\) and \(\langle \Free, F \rangle \xlongrightarrow{\Free} \langle {\bf 0}, F \rangle \)

  Similiar to above.

\item Case \( P = (*x)(P_1,P_2) \) and \( \frac{\snnull \notin F
      \andalso \scon\Sirx \in F} { \langle \Sirx(P_1, P_2), F \rangle
      \rightarrow \langle P_1, F\cup \{\snull \} \rangle } \) 

  We need to prove \(\OK_n(P_1, F\cup\{\snull\})\).  Assume that
  \(\OK_n(P_1, F\cup\{\snull\})\) does not hold. Then, we have (1) \(
  \exists \sigma \) and \(Q\) s.t. \( \langle P_1, F\cup\{\snull\}
  \rangle \xlongrightarrow{\sigma} \langle Q, F' \rangle \) and
  \(\sharp_{m}(\sigma) - \sharp_{f}(\sigma) > n\) or (2) \(
  OK(F\cup\{\snull\})\) does not hold.

  From the definition of that \(OK_n(\Sirx(P_1, P_2), F)\) holds,
  we have (1) \( \langle \scon\Sirx(P_1, P_2), F \rangle \rightarrow
  \langle P_1, F \rangle \xlongrightarrow{\sigma} \langle Q, F'
  \rangle \) s.t.  \(\sharp_m(\sigma) - \sharp_f(\sigma) \le n \),
  which is a contradiction; and (2) \(OK(F)\)
  holds. \(OK(F\cup\{\snull\})\) holds by
  Lemma~\ref{lem:okFpreserved}. Therefore, \(\OK_n(P_1,
  F\cup\{\snull\})\) holds.


\item Case \( P = \Sirx(P_1,P_2) \) and \( \frac{ \scon\Sirx \notin F}
  { \langle \Sirx(P_1, P_2), F \rangle \rightarrow \langle P_1, F
    \rangle } \)

  We need to prove \(\OK_n(P_1, F)\).  Assume that \(\OK_n(P_1, F)\)
  does not hold. Then, we have (1) \( \exists \sigma \) and \(Q\)
  s.t. \( \langle P_1, F \rangle \xlongrightarrow{\sigma} \langle Q,
  F' \rangle \) and \(\sharp_{m}(\sigma) - \sharp_{f}(\sigma) > n\) or
  (2) \( OK(F)\) does not hold.

  From the definition of that \(OK_n(\Sirx(P_1, P_2), F)\) holds, we
  have (1) \( \langle \Sirx(P_1, P_2), F \rangle \rightarrow \langle
  P_1, F \rangle \xlongrightarrow{\sigma} \langle Q, F' \rangle \)
  s.t. \(\sharp_m(\sigma) - \sharp_f(\sigma) \le n \) and (2)
  \(OK(F)\) holds, which are in contradiction to the assumption.
  Therefore, \(\OK_n(P_1, F)\) holds.

\item Case \( P = \Sirx(P_1,P_2) \) and \( \frac{ \scon\Sirx \notin F}
  { \langle \Sirx(P_1, P_2), F \rangle \rightarrow \langle P_2, F
    \rangle } \)

  We need to prove \(\OK_n(P_2, F)\).  Assume that \(\OK_n(P_2, F)\)
  does not hold. Then, we have (1) \( \exists \sigma \) and \(Q\)
  s.t. \( \langle P_2, F \rangle \xlongrightarrow{\sigma} \langle Q,
  F' \rangle \) and \(\sharp_{m}(\sigma) - \sharp_{f}(\sigma) > n\) or
  (2) \( OK(F)\) does not hold.

  From the definition of that \(OK_n(\Sirx(P_1, P_2), F)\) holds, we
  have (1) \( \langle \Sirx(P_1, P_2), F \rangle \rightarrow \langle
  P_2, F \rangle \xlongrightarrow{\sigma} \langle Q, F' \rangle \)
  s.t. \(\sharp_m(\sigma) - \sharp_f(\sigma) \le n \) and (2)
  \(OK(F)\) holds, which are in contradiction to the assumption.
  Therefore, \(\OK_n(P_2, F)\) holds.

\item Case \( P = \Sirx(P_1,P_2) \) and \( \frac{ \snull \in F
  \andalso \scon\Sirx \in F} { \langle \Sirx(P_1, P_2), F \rangle
  \rightarrow \langle P_1, F \rangle } \)

  We need to prove \(\OK_n(P_1, F)\).  Assume that \(\OK_n(P_1, F)\)
  does not hold. Then, we have (1) \( \exists \sigma \) and \(Q\)
  s.t. \( \langle P_1, F \rangle \xlongrightarrow{\sigma} \langle Q,
  F' \rangle \) and \(\sharp_{m}(\sigma) - \sharp_{f}(\sigma) > n\) or
  (2) \( OK(F)\) does not hold.

  From the definition of that \(OK_n(\Sirx(P_1, P_2), F)\) holds, we
  have (1) \( \langle \Sirx(P_1, P_2), F \rangle \rightarrow \langle
  P_2, F \rangle \xlongrightarrow{\sigma} \langle Q, F' \rangle \)
  s.t. \(\sharp_m(\sigma) - \sharp_f(\sigma) \le n \) and (2)
  \(OK(F)\) holds, which are in contradiction to the assumption.
  Therefore, \(\OK_n(P_1, F)\) holds.

\item Case \( P = \Sirx(P_1,P_2) \) and \( \frac{ \snull \in F
  \andalso \scon\Sirx \in F} { \langle \Sirx(P_1, P_2), F \rangle
  \rightarrow \langle P_1, F \rangle } \)

  We need to prove \(\OK_n(P_1, F)\).  Assume that \(\OK_n(P_1, F)\)
  does not hold. Then, we have (1) \( \exists \sigma \) and \(Q\)
  s.t. \( \langle P_1, F \rangle \xlongrightarrow{\sigma} \langle Q,
  F' \rangle \) and \(\sharp_{m}(\sigma) - \sharp_{f}(\sigma) > n\) or
  (2) \( OK(F)\) does not hold.

  From the definition of that \(OK_n(\Sirx(P_1, P_2), F)\) holds, we
  have (1) \( \langle \Sirx(P_1, P_2), F \rangle \rightarrow \langle
  P_2, F \rangle \xlongrightarrow{\sigma} \langle Q, F' \rangle \)
  s.t. \(\sharp_m(\sigma) - \sharp_f(\sigma) \le n \) and (2)
  \(OK(F)\) holds, which are in contradiction to the assumption.
  Therefore, \(\OK_n(P_1, F)\) holds.
   

\end{itemize}

\end{proof}

\begin{pfof}{Lemma~\ref{lem:preservation}}
By induction on the derivation of \(\langle H,R,s,n, C \rangle
\xlongrightarrow{\rho} \langle H',R',s', n', C \rangle\).

\begin{itemize}

\item Case: \( \langle H, R, \scon\Sirx s, n, C \rangle
  \xlongrightarrow{\Startconst} \langle H, R, s;\Endconst, n, C\cup
  {x} \rangle \)

  We have \( \Theta; \Gamma \vdash \scon\Sirx s:P\) and \( OK_n(P)
  \). From the inversion of typing rules, we get \( \Theta; \Gamma
  \vdash s:P'' \) and \( \scon\Sirx P'' \le P \) for some \( P''
  \). By subtyping, we get \( P'';\Endconst \le Q \) and \( P
  \xLongrightarrow{\Startconst} Q \) for some \( Q \).

  we need to find \(P'\) s.t. \( \Theta; \Gamma \vdash
  s;\Endconst:P'\), \( OK_n(P')\) and \( P
  \xLongrightarrow{\Startconst} P' \). Taking \( Q \) as \( P'\), \( P
  \xLongrightarrow{\Startconst}\) and \( OK_n(P')\) hold. From \(
  \Theta; \Gamma \vdash s;\Endconst:P'';\Endconst \), \( P'';\Endconst
  \le Q \) and \rn{T-Sub}, we have \( \Theta; \Gamma \vdash
  s;\Endconst:P'\).

\item Case: \( \langle H, R, \Endconst, n, C \uplus {x} \rangle
  \xlongrightarrow{\Endconst} \langle H, R, \SKIP, n, C \rangle \)

  We have \( \Theta; \Gamma \vdash \Endconst:P\) and \( OK_n(P)
  \). From the inversion of typing rules, we get \( \Theta; \Gamma
  \vdash \Endconst:\Endconst \) and \( \Endconst \le P \). By
  subtyping, we get \( 0 \le P'' \) and \( P
  \xLongrightarrow{\Endconst} P'' \) for some \( P'' \).

  we need to find \(P'\) s.t. \( \Theta; \Gamma \vdash
  \SKIP:P'\), \( OK_n(P')\) and \( P \xLongrightarrow{\Endconst} P'
  \). Taking \( P'' \) as \( P'\), \( P \xLongrightarrow{\Endconst}
  P'\) and \( OK_n(P')\) hold. From \rn{T-Skip}, \rn{T-Sub} and \( 0
  \le P''\), we have \( \Theta; \Gamma \vdash \SKIP:P'\).

\item Case: \(\langle H, R, \FREE, n, C\rangle \xlongrightarrow{\Free}
  \langle H', R', \SKIP, n + 1, C \rangle \)

We have \(\OK_n(P)\) and \(\Theta; \Gamma \vdash \Free(x) \COL P\).
From inversion of the typing rules, we have \(\Theta; \Gamma \vdash
\Free(x) \COL \Free\) and \(\Free \le P\).  Hence, from the definition
of subtyping, we have \(\TSKIP \le P''\) and \(P
\xLongrightarrow{\Free} P''\) for some \(P''\).  We need to find
\(P'\) such that \(P \xLongrightarrow{\Free} P'\), \(\Theta; \Gamma
\vdash \SKIP \COL P'\), and \(\OK_{n+1}(P')\).  Take \(P''\) as
\(P'\).  Then, \(P \xLongrightarrow{\Free} P''\) as we stated above.
We also have \(\Theta; \Gamma \vdash \SKIP \COL P''\) from
\rn{T-Skip}, \(\TSKIP \le P''\), and \rn{T-Sub}.  \(\OK_{n+1}(P'')\)
follows from Lemma~\ref{lem:okPreserved}.

\item Case: \( \langle H, R, \LET x = \MALLOC \IN s, n, C \rangle
  \xlongrightarrow{\Malloc} \langle H', R', [x'/x]s, n - 1, C \rangle \)

From the assumption, we have \(\Theta; \Gamma \vdash \LET x = \MALLOC
\IN s \COL P\) and \(\OK_{n}(P)\). By the inversion of typing rules,
we have \(\Malloc;P_1 \le P\) and \(\Theta; \Gamma \vdash s : P_{1}\)
for some \(P_1\). We have the following derivation: \infrule{ \Malloc
  \xlongrightarrow{\Malloc} 0} {\Malloc;P_1 \xlongrightarrow{\Malloc}
  0;P_{1}} and \(0;P_1 \rightarrow P_{1}\), then we have
\(\Malloc;P_{1} \xLongrightarrow{\Malloc} P_{1}\). Hence, By the
definition of subtyping, we have \(P \xLongrightarrow{\Malloc}P''\)
and \(P_{1} \le P''\) for some \(P''\).

We need to find \(P'\) and \(\Gamma'\) such that \(\Theta; \Gamma'
\vdash [x'/x]s\COL P'\) and \(P\xLongrightarrow{\Malloc} P'\). Take
\(P''\) as \(P'\) and \([x'/x]\Gamma\) as \(\Gamma'\).  (Note that
\([x'/x]\Gamma\) is well-formed because \(x'\) is fresh.)  Then \(P
\xLongrightarrow{\Malloc} P''\) as we state above. We also have
\(\Theta;[x'/x]\Gamma \vdash [x'/x]s \COL P''\) from \rn{T-Sub},
\(\Theta;[x'/x]\Gamma \vdash [x'/x]s \COL P_1\), and \(P_1 \le P''\).
\(\OK_{n-1}(P'')\) follows from Lemma~\ref{lem:okPreserved}.

\item Case: \( \langle H, R, \SKIP;s, n, C \rangle \rightarrow \langle
  H, R, s, n, C \rangle \)

  we have \(\Theta;\Gamma \vdash \SKIP;s\COL  P\) and
  \(\OK_{n}(P)\). From the inversion of the typing rules, we have
  \(\Theta; \Gamma \vdash s\COL P_{1}\) and \(0;P_{1} \le
  P\). Hence, from the definition of subtyping, we have \(P
  \xLongrightarrow{\tau} P''\) and \(P_{1} \le P''\) for some \(P''\).

  We need to find \(P'\) such that \(\Theta; \Gamma \vdash s : P'\)
  and \(P \xLongrightarrow{\tau} P'\). Take \(P''\) as \(P'\). Then \(P
  \xLongrightarrow{\tau} P''\) as we stated above. We also have
  \(\Theta;\Gamma \vdash s\COL P''\) from \rn{T-Sub}, \(\Gamma \vdash
  s\COL P_{1}\) and \(P_{1} \le P''\). \(\OK_n(P'')\) follows from
  Lemma~\ref{lem:okPreserved}

\item Case: \( \langle H, R, *x \leftarrow y , n, C \rangle \rightarrow
  \langle H', R', \SKIP, n, C \rangle \)

  We have \(\Theta; \Gamma \vdash *x \leftarrow y : P\) and
  \(\OK_{n}(P)\). From the inversion of typing rules, we have \(0 \le
  P\).

  We need to find $P'$ such that \(\Theta; \Gamma \vdash \SKIP: P'\),
  \(P \xLongrightarrow{\tau} P'\) and \(\OK_n(P')\). Take $P$ as $P'$. Then,
  \(P \xLongrightarrow{\tau} P'\) and \(\OK_n(P')\) hold. We also have \(\Theta;
  \Gamma \vdash \SKIP: P'\) from \rn{T-Skip}, \(0 \le P\) and
  \rn{T-Sub}.

\item Case: \( \langle H, R, \LET x = y\ \IN s , n, C \rangle
  \rightarrow \langle H', R', [x'/x]s, n, C \rangle \)a

  We have \(\Theta; \Gamma \vdash \LET x = y \ \IN s \COL P\) and
  \(OK_{n}(P)\). From the inversion of typing rules, we have \(\Theta;
  \Gamma \vdash s\COL P_{1}\) and \(P_{1} \le P\).

  We need to find $P'$ and \(\Gamma'\) such that \(\Theta; \Gamma'
  \vdash [x'/x]s : P'\) , \(P \xLongrightarrow{\tau} P'\) and
  \(\OK_n(P'\)). Take \(P\) as \(P'\) and \([x'/x]\Gamma\) as
  \(\Gamma'\). Then \( P \xLongrightarrow{\tau} P'\) and \(\OK_n(P')\)
  hold.  We also have \(\Theta; \Gamma \vdash [x'/x]s\COL P\) from
  \rn{T-Sub}, \(\Theta; \Gamma \vdash [x'/x]s\COL P_{1}\) and \( P_{1}
  \le P\).

\item Case: \( \langle H, R, \LET x = \NULL \ \IN \ s, n, C\rangle
  \rightarrow \langle H', R', [x'/x]s, n, C \rangle \)

  We have \(\Theta; \Gamma \vdash \LET x = \NULL \ \IN \ s\COL P\)
  and \(OK_{n}(P)\). From the inversion of typing rules, we have
  \(\Theta; \Gamma \vdash s\COL P_{1}\) and \( P_{1} \le P\).

  We need to find $P'$ and \(\Gamma'\) such that \(\Theta; \Gamma'
  \vdash [x'/x]s\COL P'\), \(P \xLongrightarrow{\tau} P'\) and
  \(\OK_n(P')\).  Take \(P\) as \(P'\) and \([x'/x]\Gamma\) as
  \(\Gamma'\).  Then, \(P \xLongrightarrow{\tau} P'\) and
  \(\OK_n(P')\) hold.  We also have \(\Theta; \Gamma \vdash
       [x'/x]s\COL P\) from \rn{T-Sub}, \(\Theta; \Gamma \vdash
       [x'/x]s\COL P_{1}\)\( P_{1} \le P\).

\item Case: \( \langle H, R, \LET x = *y \ \IN \ s, n, C\rangle
  \rightarrow \langle H', R', [x'/x]s, n, C\rangle \)

  We have \(\Theta; \Gamma \vdash \LET x = *y \ \IN \ s\COL  P\) and
  \(OK_{n}(P)\). From the inversion of typing rules, we have \(\Theta;
  \Gamma \vdash s\COL P_{1}\) and \(P_{1} \le P\).

  We need to find \(P'\) and \(\Gamma'\) such that \(\Theta; \Gamma'
  \vdash [x'/x]s\COL P'\), \(P \xLongrightarrow{\tau} P'\) and
  \(\OK_n(P')\). Take \(P\) as \(P'\) and \([x'/x]\Gamma\) as
  \(\Gamma'\). Then, \(P \xLongrightarrow{\tau} P'\) and \(\OK_n(P')\)
  hold.  We also have \(\Theta; \Gamma' \vdash [x'/x]s\COL P\) from
  \rn{T-Sub}, \(\Theta; \Gamma' \vdash [x'/x]s\COL P_{1}\) and \(P_{1}
  \le P\).

\item Case: \(\langle H, R, \IFNULL \Sirx \ \THEN s_{1} \ \ELSE
  \ s_{2}, n, C\rangle \xlongrightarrow{\sassx} \langle H, R, s_{1}, n, C \rangle\)

  We have \(\Theta; \Gamma \vdash \IFNULL \Sirx \ \THEN \ s_{1}
  \ \ELSE \ s_{2}\COL P\) and \(\OK_{n}(P)\). From the inversion of
  typing rules, we have \(\Theta; \Gamma \vdash s_{1} : P_{1}\),
  \(\Theta; \Gamma \vdash s_{2} : P_{2}\) and \( (*x)(P_1,P_2) \le
  P\). By subtyping we get \(P \xLongrightarrow{\sassx} \) \(
  P''\)and \( P_1 \le P''\) for some \( P '' \).

  We need to find $P'$ such that \(\Theta; \Gamma \vdash s_1\COL P'\)
  , \(P \xLongrightarrow{\sassx} P'\) and \(\OK_n(P'\)).  Take \(P''\) as
  \(P'\).  Then, \(\OK_n(P')\) and \(P \xLongrightarrow{\sassx} P'\)
  hold. We also have \(\Theta; \Gamma' \vdash s_{1}\COL P'\) from
  \(\Theta; \Gamma \vdash s_{1}\COL P_1\), \(P_{1} \le P'' \) and
  \rn{T-Sub}.

\item Case: \(\langle H, R, \IFNULL \Sirx \ \THEN s_{1} \ \ELSE \ s_{2},
  n , C \rangle \xlongrightarrow{\sassxn} \langle H, R, s_{2}, n, C\rangle\).

 The proof is similar to case of  \(\langle H, R, \IFNULL \Sirx \ \THEN s_{1} \ \ELSE
  \ s_{2}, n, C \rangle \xlongrightarrow{\sassx} \langle H, R, s_{1}, n, C\rangle\).

\item Case: \( \langle H, R, f(\vec{x}) , n, C \rangle \rightarrow  \langle H, R, [\vec{x}/\vec{y}]s, n, C \rangle \)

% We have \(D(f) = s\) and \(\Theta(f) = P_1\), we have \(s \COL P_1\). 

From the inversion of \rn{T-Call}, we have \(\Theta; \Gamma \vdash
f(\vec{x}) \COL \Theta(f)\) and \(\Theta(f) \le P\).  From
\rn{Sem-Call}, we have \(D(f) = (\vec{y})s\).  From the assumption
\(\vdash D \COL \Theta\), we have \(\Theta; \vec{y} \vdash s \COL
\Theta(f)\); hence, with \(\Theta(f) \le P\), we have \(\Theta;
\vec{y} \vdash s \COL P\).  We then have \(\Theta; \vec{x} \vdash
    [\vec{x}/\vec{y}]s \COL P\) as required.

\end{itemize}
\end{pfof}  

\begin{corollary}
\label{cor:preservation}
If $OK_{n}(P)$, $\Theta; \Gamma \vdash s : P$, \(\vdash D \COL
\Theta\), and $\langle H,R,s,n,C\rangle \xlongrightarrow{\sigma}
\langle H',R',s', n', C \rangle$, then there exists $P'$ such that (1) $
\Theta; \Gamma \vdash s' : P'$, (2) \(P \xLongrightarrow{\sigma} P'\),
and (3) \(OK_{n'}(P')\).
\end{corollary}

We write \(\langle H, R, s, n \rangle \xlongrightarrow{\rho}\) if
there is a transition \(\xlongrightarrow{\rho}\) from \(\langle H, R,
s, n \rangle\).

\begin{lemma}
\label{lem:enabled}
If \(\Theta; \Gamma \vdash s \COL P\) and \(\langle H, R, s, n \rangle
\xlongrightarrow{\rho}\) and \(\rho \in \set{\Malloc, \Free, \sassx, \sassxn, \Startconst, \Endconst}\), then
there exists \(P'\) such that \(P \xLongrightarrow{\rho} P'\).
\end{lemma}

\begin{proof}
Induction on the derivation of \(\Theta; \Gamma \vdash s \COL P\).
\end{proof}

\begin{pfof}{Lemma~\ref{lem:immediateSafety}}

By contradiction.  Assume \(\langle H, R, s, n \rangle
\xlongrightarrow{\rho} \OVERFLOW\). Then, \(n\) is \(0\) and \(\rho =
\Malloc\) from \rn{Sem-OutOfMem}.  From the assumption \(\Theta;\Gamma
\vdash s \COL P\) and \(\OK_0(P)\).  From Lemma~\ref{lem:enabled},
there exists \(P'\) such that \(P \xLongrightarrow{\Malloc} P'\).
However, this contradicts \(\OK_0(P)\).


\end{pfof}


\begin{pfof}{Theorem~\ref{thm1}}

We have \(\Theta;\emptyset \vdash s\COL P, \vdash D\COL \Theta\) and
\(\OK_n(P)\).

Suppose that there exists \(\sigma\) such that \(\langle \emptyset,
\emptyset, s, n\rangle \xlongrightarrow{\sigma} \langle H', R', s',
n'\rangle \xlongrightarrow{\rho} \OVERFLOW\).  Then, \(n' = 0\) and
\(\rho = \Malloc\).  From Lemma~\ref{cor:preservation}, there exists
\(P'\) such that \(\Theta; \Gamma' \vdash s \COL P'\), \(P
\xLongrightarrow{\sigma} P'\), and \(\OK_0(P')\); hence \(\langle H',
R', s', 0\rangle \xlongrightarrow{\Malloc}\).  However, this
contradicts Lemma~\ref{lem:immediateSafety}.

\end{pfof}
