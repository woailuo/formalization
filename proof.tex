\section{Proof of Lemmas}
\label{sec:proof}

\begin{lemma}
\label{lem:okFpreserved}
If \( \langle P, F \rangle \xlongrightarrow{\rho} \langle P', F' \rangle\) and \( OK(F)\), then \(OK(F')\)
\end{lemma}

\begin{proof}
By induction on \( \langle P, F \rangle \xlongrightarrow{\rho} \langle P', F' \rangle\).
  \begin{itemize}

  \item Case \( P = {\bf 0};P'\) and \( \langle {\bf 0};P', F \rangle \rightarrow \langle P', F \rangle \)
    
      We need to prove \(OK(F')\). From assumption, we have that
      \(OK(F)\) holds, and in this case F' is the same as
      F. Therefore, \(OK(F')\) holds.

    \item Case \( P = \Malloc\) and \( \langle \Malloc, F \rangle \xlongrightarrow{\Malloc} \langle {\bf 0}, F \rangle \) 

      Similiar to above.

    \item Case \( P = \Free\) and \( \langle \Free, F \rangle \xlongrightarrow{\Free} \langle {\bf 0}, F \rangle \)

      Similiar to above.

    \item Case \( P = \Sirx(P_1,P_2)\) and \( \frac{\snnull \notin F
      \andalso \scon\Sirx \in F} { \langle \Sirx(P_1, P_2), F \rangle
      \rightarrow \langle P_1, F\cup \{\snull \} \rangle } \)
      
      We need to prove \( OK(F\cup\{ \snull\})\). From the assumption,
      we have that \(OK(F)\) holds and \( \snnull \notin F\). Then, we
      have \(F\cup\{ \snull\}\) does not contain both \(\snull\) and
      \( \snnull\). Therefore, \( OK(F\cup\{ \snull\})\) holds.

    \item Case \( P = \Sirx(P_1,P_2)\) and \( \frac{\scon\Sirx \notin F}
      { \langle \Sirx(P_1, P_2), F \rangle
      \rightarrow \langle P_1, F \rangle } \) \\
      We need to prove \( OK(F)\). From the assumption, \(OK(F)\) holds.

    \item Case \( P = \Sirx(P_1,P_2)\) and \( \frac{\scon\Sirx \notin F}
      { \langle \Sirx(P_1, P_2), F \rangle
      \rightarrow \langle P_2, F \rangle } \) \\
      We need to prove \( OK(F)\). From the assumption, \(OK(F)\) holds.

    \item Case \( P = \Sirx(P_1,P_2)\) and \( \frac{ \snull \in F \andalso \scon\Sirx \in F}
      { \langle \Sirx(P_1, P_2), F \rangle
      \rightarrow \langle P_1, F \rangle } \) \\
      We need to prove \( OK(F)\). From the assumption, \(OK(F)\) holds.

    \item Case \( P = \scon\Sirx P'\) and \(  \langle \scon\Sirx
      P', F \rangle \rightarrow \langle P';\Endconst, F\cup\{\scon\Sirx\} \rangle \) \\
      We need to prove \( OK(F\cup\{\scon\Sirx\})\). From the
      assumption, we have \(OK(F)\) holds. Also,
      \(F\cup\{\scon\Sirx\}\) does not contain both \(\snull\) and \(
      \snnull\). Therefore,\( OK(F\cup\{ \scon\Sirx\})\) holds.

    \item Case \( P = \Sirx(P_1,P_2)\) and \( \frac{ \snull \notin F \andalso \scon\Sirx \notin F}
      { \langle \Sirx(P_1, P_2), F \rangle
      \rightarrow \langle P_2, F\cup\{\snnull\} \rangle } \) \\
      We need to prove \( OK(F\cup\{\snnull\})\). From the assumption,
      we have that \(OK(F)\) holds and \( \snull \notin F\). Then we
      get \(F\cup\{\snnull\}\) does not contain both \(\snull\) and
      \(\snnull\). Therefore, \( OK(F\cup\{\snnull\})\) holds.


    \item Case \( P = \mu\alpha.P'\) and \( \langle \mu\alpha.P', F \rangle
      \rightarrow \langle [\mu\alpha.P']P', F \rangle \) \\
      We need to prove \( OK(F)\). From the assumption, we have that
      \(OK(F)\) holds.

    \item Case \( P = P_1 + P_2\) and \( \langle P_1 + P_2, F \rangle
      \rightarrow \langle P_1, F \rangle \) \\
      We need to prove \( OK(F)\). From the assumption, we have that
      \(OK(F)\) holds.

    \item Case \( P = P_1 + P_2\) and \(  \langle P_1 + P_2, F \rangle
      \rightarrow \langle P_2, F \rangle \) \\
      We need to prove \( OK(F)\). From the assumption, we have that
      \(OK(F)\) holds.

    \item Case \( P = P_1;P_2\) and \( \frac{ \langle P_1, F \rangle \xlongrightarrow{\rho} \langle P_1', F' \rangle }
      {\langle P_1;P_2, F \rangle \xlongrightarrow{\rho} \langle P_1';P_2, F' \rangle} \) \\
    We need to prove \(OK(F')\). By IH, we have \( \langle P_1, F
    \rangle \xlongrightarrow{\rho} \langle P_1', F' \rangle \) and \(
    OK(F) \) holds, then \(OK(F')\) holds.
      
      
  \end{itemize}
\end{proof}



\begin{lemma}
\label{lem:okPreserved}
If \(\OK_n(P, F)\) and \( \langle P, F \rangle \xlongrightarrow{\rho} \langle P', F' \rangle\), then
\begin{itemize}
\item \(\OK_{n-1}(P', F')\) if \(\rho = \Malloc\),
\item \(\OK_{n+1}(P', F')\) if \(\rho = \Free\),
\item \(\OK_n(P', F')\) if \(\rho = \mbox{Otherwise}  \)
\end{itemize}
\end{lemma}

\begin{proof}
By induction on \(\langle P, F \rangle \xlongrightarrow{\rho} \langle P', F' \rangle \).

\begin{itemize}

\item Case \(P = {\bf 0};P'\) and \( \langle {\bf 0};P', F \rangle \rightarrow \langle P', F \rangle\)

  We need to prove \(\OK_n(P', F)\).  Assume that \(\OK_n(P', F)\)
  does not hold. Then, we have (1) \( \exists \sigma \) and \(Q\) s.t. \(
  \langle P', F \rangle \xlongrightarrow{\sigma} \langle Q, F' \rangle
  \), \(\sharp_{m}(\sigma) - \sharp_{f}(\sigma) > n\) or (2) \(
  OK(F)\) does not hold.

  From the definition of that \(OK({\bf 0};P', F)\) holds, we have (1)
  if \( \langle {\bf 0};P', F \rangle \rightarrow \langle P', F
  \rangle \xlongrightarrow{\sigma} \langle Q, F' \rangle \), then
  \(\sharp_m(\sigma) - \sharp_f(\sigma) \le n \) and (2) \(OK(F)\),
  which are in contradiction to the assumption. Therefore, \(\OK_n(P',
  F)\) holds.

\item Case \(P = \Malloc\) and \( \langle \Malloc, F \rangle \xlongrightarrow{\Malloc} \langle {\bf 0}, F \rangle\)

  We need to prove \(\OK_{n-1}({\bf 0}, F)\), which means we need to
  prove (1) \( \forall \sigma \) and \(Q\) if \( \langle {\bf 0}, F
  \rangle \xlongrightarrow{\sigma} \langle Q, F' \rangle \), then \(\sharp_{m}(\sigma) - \sharp_{f}(\sigma) \le n\)
  and (2) \( OK(F)\) holds.  There is no \(Q\) and \(\sigma\)
  s.t. \(\langle {\bf 0}, F \rangle \xlongrightarrow{\sigma} \langle
  Q, F \rangle \), so (1) holds.  \(OK(F)\) holds from
  Lemma~\ref{lem:okFpreserved} . Therefore, \(OK({\bf 0}, F)\) holds.
  
\item Case \(P = \Free\) and \(\langle \Free, F \rangle \xlongrightarrow{\Free} \langle {\bf 0}, F \rangle \)

  Similiar to above.

\item Case \( P = (*x)(P_1,P_2) \) and \( \frac{\snnull \notin F
      \andalso \scon\Sirx \in F} { \langle \Sirx(P_1, P_2), F \rangle
      \rightarrow \langle P_1, F\cup \{\snull \} \rangle } \) 

  We need to prove \(\OK_n(P_1, F\cup\{\snull\})\).  Assume that
  \(\OK_n(P_1, F\cup\{\snull\})\) does not hold. Then, we have (1) \(
  \exists \sigma \) and \(Q\) s.t. \( \langle P_1, F\cup\{\snull\}
  \rangle \xlongrightarrow{\sigma} \langle Q, F' \rangle \) and \(\sharp_{m}(\sigma) - \sharp_{f}(\sigma) > n\) or
  (2) \( OK(F\cup\{\snull\})\) does not hold.

  From the definition of that \(OK_n(\Sirx(P_1, P_2), F)\) holds, we
  have (1) if \( \langle \scon\Sirx(P_1, P_2), F \rangle \rightarrow
  \langle P_1, F \rangle \xlongrightarrow{\sigma} \langle Q, F'
  \rangle \), then \(\sharp_m(\sigma) -
  \sharp_f(\sigma) \le n \), which is a contradiction; and (2)
  \(OK(F)\) holds. \(OK(F\cup\{\snull\})\) holds by
  Lemma~\ref{lem:okFpreserved}. Therefore, \(\OK_n(P_1,
  F\cup\{\snull\})\) holds.


\item Case \( P = \Sirx(P_1,P_2) \) and \( \frac{ \scon\Sirx \notin F}
  { \langle \Sirx(P_1, P_2), F \rangle \rightarrow \langle P_1, F
    \rangle } \)

  We need to prove \(\OK_n(P_1, F)\).  Assume that \(\OK_n(P_1, F)\)
  does not hold. Then, we have (1) \( \exists \sigma \) and \(Q\)
  s.t. \( \langle P_1, F \rangle \xlongrightarrow{\sigma} \langle Q,
  F' \rangle \) and \(\sharp_{m}(\sigma) -
  \sharp_{f}(\sigma) > n\) or (2) \( OK(F)\) does not hold.

  From the definition of that \(OK_n(\Sirx(P_1, P_2), F)\) holds, we
  have (1) if \( \langle \Sirx(P_1, P_2), F \rangle \rightarrow
  \langle P_1, F \rangle \xlongrightarrow{\sigma} \langle Q, F'
  \rangle \) then \(\sharp_m(\sigma) -
  \sharp_f(\sigma) \le n \) and (2) \(OK(F)\) holds, which are in
  contradiction to the assumption.  Therefore, \(\OK_n(P_1, F)\)
  holds.

\item Case \( P = \Sirx(P_1,P_2) \) and \( \frac{ \scon\Sirx \notin F}
  { \langle \Sirx(P_1, P_2), F \rangle \rightarrow \langle P_2, F
    \rangle } \)

  We need to prove \(\OK_n(P_2, F)\).  Assume that \(\OK_n(P_2, F)\)
  does not hold. Then, we have (1) \( \exists \sigma \) and \(Q\)
  s.t. \( \langle P_2, F \rangle \xlongrightarrow{\sigma} \langle Q,
  F' \rangle \) and \(\sharp_{m}(\sigma) -
  \sharp_{f}(\sigma) > n\) or (2) \( OK(F)\) does not hold.

  From the definition of that \(OK_n(\Sirx(P_1, P_2), F)\) holds, we
  have (1) if \( \langle \Sirx(P_1, P_2), F \rangle \rightarrow
  \langle P_2, F \rangle \xlongrightarrow{\sigma} \langle Q, F'
  \rangle \), then \(\sharp_m(\sigma) -
  \sharp_f(\sigma) \le n \) and (2) \(OK(F)\) holds, which are in
  contradiction to the assumption.  Therefore, \(\OK_n(P_2, F)\)
  holds.

\item Case \( P = \Sirx(P_1,P_2) \) and \( \frac{ \snull \in F
  \andalso \scon\Sirx \in F} { \langle \Sirx(P_1, P_2), F \rangle
  \rightarrow \langle P_1, F \rangle } \)

  We need to prove \(\OK_n(P_1, F)\).  Assume that \(\OK_n(P_1, F)\)
  does not hold. Then, we have (1) \( \exists \sigma \) and \(Q\)
  s.t. \( \langle P_1, F \rangle \xlongrightarrow{\sigma} \langle Q,
  F' \rangle \) and \(\sharp_{m}(\sigma) -
  \sharp_{f}(\sigma) > n\) or (2) \( OK(F)\) does not hold.

  From the definition of that \(OK_n(\Sirx(P_1, P_2), F)\) holds, we
  have (1) if \( \langle \Sirx(P_1, P_2), F \rangle \rightarrow
  \langle P_2, F \rangle \xlongrightarrow{\sigma} \langle Q, F'
  \rangle \), then \(\sharp_m(\sigma) -
  \sharp_f(\sigma) \le n \) and (2) \(OK(F)\) holds, which are in
  contradiction to the assumption.  Therefore, \(\OK_n(P_1, F)\)
  holds.

\item Case \( P = \scon\Sirx P' \) and \( \langle \scon\Sirx P', F \rangle
  \rightarrow \langle P';\Endconst, F\cup{\scon\Sirx} \rangle  \)

  We need to prove \(\OK_n(P';\Endconst, F\cup{\scon\Sirx})\).  Assume
  that \(\OK_n(P';\Endconst, F\cup{\scon\Sirx})\) does not hold. Then,
  we have (1) \( \exists \sigma \) and \(Q\) s.t. \( \langle
  P';\Endconst, F\cup{\scon\Sirx} \rangle \xlongrightarrow{\sigma}
  \langle Q, F' \rangle \) and \(\sharp_{m}(\sigma) -
  \sharp_{f}(\sigma) > n\) or (2) \( OK(F\cup{\scon\Sirx})\) does not
  hold.

  From the definition of that \(OK_n(\scon\Sirx P', F)\) holds, we
  have (1) if \( \langle \scon\Sirx P', F \rangle \rightarrow \langle
  P;\Endconst, F\cup{\scon\Sirx} \rangle \xlongrightarrow{\sigma}
  \langle Q, F' \rangle \), then \(\sharp_m(\sigma) -
  \sharp_f(\sigma) \le n \) and (2) \(OK(F)\) holds, which are in
  contradiction to the assumption.  Therefore, \(\OK_n(P_1, F)\)
  holds.

\item Case \( P = \Sirx(P_1,P_2) \) and \( \frac{ \snull \notin F
  \andalso \scon\Sirx \in F} { \langle \Sirx(P_1, P_2), F \rangle
  \rightarrow \langle P_2, F\cup{\snnull} \rangle } \)

  We need to prove \(\OK_n(P_2, F\cup{\snnull})\).  Assume that
  \(\OK_n(P_2, F\cup{\snnull})\) does not hold. Then, we have (1) \(
  \exists \sigma \) and \(Q\) s.t. \( \langle P_2, F\cup{\snnull}
  \rangle \xlongrightarrow{\sigma} \langle Q, F' \rangle \) and
  \(\sharp_{m}(\sigma) - \sharp_{f}(\sigma) > n\) or (2) \(
  OK(F\cup{\snnull})\) does not hold.

  From the definition of that \(OK_n(\Sirx(P_1, P_2), F)\) holds, we
  have (1) if \( \langle \Sirx(P_1, P_2), F \rangle \rightarrow
  \langle P_2, F\cup{\snnull} \rangle \xlongrightarrow{\sigma} \langle
  Q, F' \rangle \), then \(\sharp_m(\sigma) -
  \sharp_f(\sigma) \le n \), which is a contradiction; and (2)
  \(OK(F)\) holds. From the Lemma~\ref{lem:okFpreserved},
  \(OK(F\cup{\snnull})\) holds. Therefore, \(OK(P_2, F\cup{\snnull})\)
  holds.

\item Case \( P = \mu\alpha.P' \) and \(  \langle \mu\alpha.P', F \rangle
  \rightarrow \langle [\mu\alpha.P'/\alpha]P', F  \rangle  \)

  We need to prove \(\OK_n([\mu\alpha.P'/\alpha]P', F) \).  Assume
  that \(\OK_n( [\mu\alpha.P'/\alpha]P', F) \) does not hold. Then, we
  have (1) \( \exists \sigma \) and \(Q\) s.t. \( \langle
  [\mu\alpha.P'/\alpha]P', F \rangle \xlongrightarrow{\sigma} \langle
  Q, F' \rangle \) and \(\sharp_{m}(\sigma) -
  \sharp_{f}(\sigma) > n\) or (2) \( OK(F)\) does not hold.

  From the definition of that \(OK_n(\mu\alpha.P', F)\) holds, we have
  (1) if \( \langle \mu\alpha.P', F \rangle \rightarrow \langle
  [\mu\alpha.P'/\alpha]P', F \rangle \xlongrightarrow{\sigma} \langle
  Q, F' \rangle \), then \(\sharp_m(\sigma) -
  \sharp_f(\sigma) \le n \), which is a contradiction; and (2)
  \(OK(F)\) holds. From the Lemma~\ref{lem:okFpreserved},
  \(OK(F\cup{\snnull})\) holds. Therefore,
  \(OK([\mu\alpha.P'/\alpha]P', F) \) holds.


\item Case \( P = P_1 + P_2 \) and \(  \langle P_1 + P_2, F \rangle
  \rightarrow \langle P_1, F  \rangle  \)

  We need to prove \(\OK_n(P_1, F) \).  Assume that \(\OK_n(P_1, F) \)
  does not hold. Then, we have (1) \( \exists \sigma \) and \(Q\)
  s.t. \( \langle P_1, F \rangle \xlongrightarrow{\sigma} \langle Q,
  F' \rangle \) and \(\sharp_{m}(\sigma) -
  \sharp_{f}(\sigma) > n\) or (2) \( OK(F)\) does not hold.

  From the definition of that \(OK_n(P_1 + P_2, F)\) holds, we have
  (1) if \( \langle P_1 + P_2, F \rangle \rightarrow \langle P_1, F
  \rangle \xlongrightarrow{\sigma} \langle Q, F' \rangle \), then \(\sharp_m(\sigma) - \sharp_f(\sigma) \le n \) and
  (2) \(OK(F)\) holds, which are in contradiction to assumption
  . Therefore, \(OK(P_1, F) \) holds.

\item Case \( P = P_1;P_2 \) and \( \frac{ \langle P_1, F \rangle
  \xLongrightarrow{\rho} \langle P_1', F' \rangle} { \langle P_1;P_2,
  F \rangle \xLongrightarrow{\rho} \langle P_1';P_2, F' \rangle } \)

  We need to prove \(\OK_{n'}(P_1';P_2, F) \), where \(n'\) is
  determined by 
  \[
   n'=\left\{
   \begin{array}{ll}
     n + 1 & \rho = \Free\\
     n - 1 & \rho = \Malloc\\
     n & \mbox{Otherwise.}
   \end{array}
   \right.
   \]

  Assume that \(\OK_{n'}(P_1';P_2, F') \) does not hold. Then, we have
  (1) \( \exists \sigma \), \(Q\) and \(F''\) s.t. \( \langle
  P_1';P_2, F \rangle \xlongrightarrow{\sigma} \langle Q, F'' \rangle
  \) and \(\sharp_{m}(\sigma) - \sharp_{f}(\sigma) > n'\)
  or (2) \( OK(F')\) does not hold.

  From the definition of that \(OK_n(P_1;P_2, F)\) holds, we have (1)
  if \( \langle P_1;P_2, F \rangle \xLongrightarrow{\rho} \langle
  P_1';P_2, F' \rangle \xlongrightarrow{\sigma} \langle Q, F'' \rangle
  \), then \(\sharp_m(\rho\sigma) -
  \sharp_f(\rho\sigma) \le n \) and (2) \(OK(F)\) holds.

  From (1), we get \( n' + \sharp_m(\rho) - \sharp_f(\rho) <
  \sharp_m(\rho) + \sharp_m(\sigma) - \sharp_f(\rho) -
  \sharp_f(\sigma) \le n\). For any \(\rho\), the \( n' +
  \sharp_m(\rho) - \sharp_f(\rho) = n\), therefore we get a
  contradiction. By IH, we have \(OK(F')\) holds, which is a
  contradiction. Therefore, \(OK_{n'}(P_1;P_2, F')\) holds.

\end{itemize}
\end{proof}

\begin{pfof}{Lemma~\ref{lem:preservation}}
By induction on the derivation of \(\langle H, R, s, n, C \rangle
\xlongrightarrow{\rho} \langle H' ,R' ,s', n', C' \rangle\).

\begin{itemize}

\item Case: \( \langle H, R, \scon\Sirx s, n, C \rangle
  \rightarrow \langle H, R, s;\Endconst, n, C\cup
  \{\scon\Sirx\} \rangle \)

  From the assumption \( \Theta; \Gamma \vdash \langle H, R,
  \scon\Sirx s, n, C \rangle : \langle P, F \rangle\), we have \(
  \Theta; \Gamma \vdash \scon\Sirx s: P \) and \( OK_n(P, F) \). From
  the inversion of typing rules, we get \( \Theta; \Gamma \vdash s:P''
  \) and \( \scon\Sirx P'' \le P \) for some \( P'' \). By subtyping,
  we have \( P'';\Endconst \le Q \) and \( \langle P, F \rangle
  \Longrightarrow \langle Q, F\cup \{\scon\Sirx\} \rangle \) for some
  \( Q \).

  we need to find \(P'\) and \(F'\) s.t. \( \Theta; \Gamma \vdash
  s;\Endconst:P'\), \( OK_n(P', F')\) and \( \langle P, F' \rangle
  \Longrightarrow \langle P', F' \rangle \). Taking \( Q \) as \( P'\)
  and \( F \cup \{\scon\Sirx\} \) as \(F'\). Therefore \( \langle P, F
  \rangle \rightarrow \langle P', F' \rangle\) holds, and \( OK_n(P',
  F')\) holds from Lemma~\ref{lem:okPreserved}. From \( \Theta; \Gamma
  \vdash s;\Endconst:P'';\Endconst \), \( P'';\Endconst \le Q \) and
  \rn{T-Sub}, \( \Theta; \Gamma \vdash s;\Endconst:P'\) holds.

\item Case: \( \langle H, R, \Endconst, n, C \rangle \rightarrow
  \langle H, R, \SKIP, n, C' \rangle \) where C' = \(filter\_C(C,
  *x)\)

   From the assumption \( \Theta; \Gamma \vdash \langle H, R,
   \Endconst, n, C \rangle : \langle P, F \rangle\), we have \(
   \Theta; \Gamma \vdash \Endconst \COL P\) and \( OK_n(P, F) \). From
   the inversion of typing rules, we get \( \Theta; \Gamma \vdash
   \Endconst:\Endconst \) and \( \Endconst \le P \). By subtyping and
   function \(filter\_T(F, *x)\), we get \( 0 \le Q \) and \( \langle
   P, F \rangle \rightarrow \langle Q, F'' \rangle\) for some \( Q \).

   we need to find \(P'\) and \(F'\) s.t. \( \Theta; \Gamma \vdash
  \SKIP:P'\), \( OK_n(P', F')\) and \( \langle P, F \rangle
  \Longrightarrow P', F' \rangle \). Taking \( Q \) as \( P'\) and
  \(F''\) as \(F'\) therefore \(F' \thickapprox C'\) from functions
  \(filter\_T(F, *x)\) and \(filter\_C(C, *x)\); \( \langle P, F
  \rangle \rightarrow \langle P', F' \rangle\) and \( OK_n(P', F')\)
  hold. From \rn{T-Skip}, \rn{T-Sub} and \( 0 \le Q\), then \( \Theta;
  \Gamma \vdash \SKIP:P'\) holds.

\item Case: \(\langle H, R, \FREE, n, C\rangle \xlongrightarrow{\Free}
  \langle H', R, \SKIP, n + 1, C \rangle \)

  From the assumption \( \Theta; \Gamma \vdash \langle H, R, \FREE, n,
  C \rangle : \langle P, F \rangle\), we have \(\OK_n(P, F)\) and
  \(\Theta; \Gamma \vdash \Free(x) \COL P\).  From inversion of the
  typing rules, we have \(\Theta; \Gamma \vdash \Free(x) \COL \Free\)
  and \(\Free \le P\). By the subtyping, we have \( \langle P, F
  \rangle \xLongrightarrow{\Free} \langle Q, F \rangle\) and \(\TSKIP
  \le Q \)for some \(Q\).

  We need to find \(P'\) and \(F'\) such that \( \langle P, F \rangle
  \xLongrightarrow{\Free} \langle P', F' \rangle \), \(\Theta; \Gamma
  \vdash \SKIP \COL P'\), and \(\OK_{n+1}(P', F')\).  Take \(Q\) as
  \(P'\) and \(F\) as \(F'\).  Then, \( \langle P, F \rangle
  \xLongrightarrow{\Free} \langle P', F' \rangle \) holds, and
  \(OK_{n+1}(P', F')\) holds from Lemma~\ref{lem:okPreserved}.  We
  also have \(\Theta; \Gamma \vdash \SKIP \COL P'\) from \rn{T-Skip},
  \(\TSKIP \le Q\) and \rn{T-Sub}.

\item Case: \( \langle H, R, \LET x = \MALLOC \IN s, n, C\rangle
  \xlongrightarrow{\Malloc} \langle H', R', [x'/x]s, n - 1, C \rangle \)

  From the assumption \( \Theta; \Gamma \vdash \langle H, R, \LET x =
  \MALLOC \IN s, n, C \rangle : \langle P, F \rangle\), we have
  \(\Theta; \Gamma \vdash \LET x = \MALLOC \IN s \COL P\) and
  \(\OK_{n}(P, F)\). By the inversion of typing rules, we have
  \(\Malloc;P'' \le P\) and \(\Theta; \Gamma \vdash s : P'' \) for
  some \(P''\). By subtyping, we get \( \langle P, F \rangle
  \xLongrightarrow{\Malloc} \langle Q, F \rangle \) and \( P'' \le
  Q\) for some \(Q\).

  We need to find \(P'\) and \(F'\) such that \(\Theta; [x'/x]\Gamma
  \vdash [x'/x]s\COL P'\) and \( \langle P, F \rangle
  \xLongrightarrow{\Malloc} \langle P', F' \rangle\) and
  \(OK_{n-1}(P', F')\). Take \(Q\) as \(P'\) and \(F\) as \(F'\).
  Then \( \langle P, F \rangle \xLongrightarrow{\Malloc} \langle P',
  F' \rangle\) holds, and \(OK_{n-1}(P', F')\) holds from
  Lemma~\ref{lem:okPreserved}. We also have \(\Theta;[x'/x]\Gamma
  \vdash [x'/x]s \COL P'\) from \rn{T-Sub}, \(\Theta;[x'/x]\Gamma
  \vdash [x'/x]s \COL P''\), and \(P'' \le Q\).
      
\item Case: \( \langle H, R, \SKIP;s, n, C \rangle \rightarrow \langle
  H, R, s, n, C \rangle \)

  From the assumption \( \Theta; \Gamma \vdash \langle H, R, \SKIP;s, n,
  C \rangle : \langle P, F \rangle\), we have \(\Theta;\Gamma
  \vdash \SKIP;s\COL P\) and \(\OK_{n}(P, F)\). From the inversion of
  the typing rules, we get \(\Theta; \Gamma \vdash s\COL P''\) and
  \(0;P'' \le P\). From the definition of subtyping, we have \(
  \langle P, F \rangle \Longrightarrow \langle Q, F \rangle\) and \(P''
  \le Q\) for some \(Q\).

  We need to find \(P'\) and \(F'\) such that \(\Theta; \Gamma \vdash s : P'\)
  and \(\langle P, F \rangle \rightarrow \langle P', F' \rangle\)
  and \(OK_n(P', F')\). Take \(Q\) as \(P'\) and \(F\) as \(F'\). Then
  \(\langle P, F\rangle \Longrightarrow \langle P', F' \rangle\) and
  \(OK_n(P', F')\) hold. We also have \(\Theta;\Gamma \vdash s\COL
  P'\) from \rn{T-Sub}, \(\Gamma \vdash s\COL P''\) and \(P'' \le Q\).

\item Case: \( \langle H, R, *x \leftarrow y , n, C\rangle \rightarrow
  \langle H', R, \SKIP, n, C\rangle \)

  From the assumption \( \Theta; \Gamma \vdash \langle H, R, *x
  \leftarrow y, n, C \rangle : \langle P, F \rangle\), we have
  \(\Theta; \Gamma \vdash *x \leftarrow y : P\) and \(\OK_{n}(P,
  F)\). From the inversion of typing rules, we have \(0 \le P\).

  We need to find \(P'\) such that \(\Theta; \Gamma \vdash \SKIP:
  P'\), \( \langle P, F \rangle \Longrightarrow \langle P', F'
  \rangle \) and \(\OK_n(P', F')\). Take \(P\) as \(P'\) and \(F\) as
  \(F'\). Then, \( \langle P, F\rangle \Longrightarrow \langle P',
  F'\rangle\) and \(\OK_n(P', F')\) hold. We also have \(\Theta;
  \Gamma \vdash \SKIP: P'\) from \rn{T-Skip}, \(0 \le P\) and
  \rn{T-Sub}.

\item Case: \( \langle H, R, \LET x = y\ \IN s , n, C \rangle
  \rightarrow \langle H, R', [x'/x]s, n, C \rangle \)

  From the assumption \( \Theta; \Gamma \vdash \langle H, R, \LET x =
  y\ \IN s, n, C \rangle : \langle P, F \rangle\), we have \(\Theta;
  \Gamma \vdash \LET x = y \ \IN s \COL P\) and \(OK_{n}(P, F)\). From
  the inversion of typing rules, we have \(\Theta; \Gamma \vdash s\COL
  P''\) and \(P'' \le P\).

  We need to find \(P'\) and \(\Gamma'\) such that \(\Theta;
  [x'/x]\Gamma \vdash [x'/x]s : P'\) , \( \langle P, F \rangle
  \Longrightarrow \langle P', F' \rangle\) and \(\OK_n(P', F'\)). Take
  \(P\) as \(P'\) and \(F\) as \(F'\). Then \( \langle P,F \rangle
  \Longrightarrow \langle P', F' \rangle\) and \(\OK_n(P', F')\) hold.  We
  also have \(\Theta; [x'/x]\Gamma \vdash [x'/x]s\COL P'\) from
  \rn{T-Sub}, \(\Theta; [x'/x]\Gamma \vdash [x'/x]s\COL P''\) and \(
  P'' \le P\).

\item Case: \( \langle H, R, \LET x = \NULL \ \IN \ s, n\rangle
  \rightarrow \langle H, R', [x'/x]s, n \rangle \)

  Similar to the above.

\item Case: \( \langle H, R, \LET x = *y \ \IN \ s, n\rangle
  \rightarrow \langle H, R', [x'/x]s, n\rangle \)

  Similar to the above.

\item Case: \(\langle H, R, \IFNULL \Sirx \ \THEN s_{1} \ \ELSE
  \ s_{2}, n, C\rangle \rightarrow \langle H, R, s_{1}, n, C \rangle\)
  if \(H(R(x)) = \NULL\) and \(\scon\Sirx \notin C\)

  From assumption \( \Theta; \Gamma \vdash \langle H, R, \IFNULL\Sirx
  \ \THEN s_{1} \ \ELSE \ s_{2}, n, C \rangle : \langle P, F
  \rangle\), we have \(\Theta; \Gamma \vdash \IFNULL\Sirx \ \THEN
  s_{1} \ \ELSE \ s_{2} \COL P \) and \(OK_n(P, F)\). From the
  inversion of typing rules, we have \(\Theta; \Gamma \vdash s_{1}
  \COL P_1\), \(\Theta; \Gamma \vdash s_{2} \COL P_2\) and \((*x)(P_1,
  P_2) \le P\). By subtyping and \(\scon\Sirx \notin C\), which means
  \(\scon\Sirx \notin F\), we get \(\langle P, F \rangle \Longrightarrow
  \langle Q, F \rangle \) and \(P_1 \le Q\) for some \(Q\).

  We need to find \(P'\) and \(F'\) such that \(\Theta;\Gamma \vdash
  s_1\COL P'\), \( \langle P, F \rangle \Longrightarrow \langle P', F'
  \rangle\) and \(\OK_n(P', F'\)). Take \(Q\) as \(P'\) and \(F\) as
  \(F'\). Then \( \langle P,F \rangle \rightarrow \langle P', F'
  \rangle\) and \(\OK_n(P', F')\) hold.  We also have \(\Theta; \Gamma
  \vdash s_1 \COL P'\) from \rn{T-Sub}, \(\Theta; \Gamma \vdash s_1
  \COL P_1\) and \( P_1 \le Q\).

\item Case: \(\langle H, R, \IFNULL \Sirx \ \THEN s_{1} \ \ELSE
  \ s_{2}, n, C\rangle \rightarrow \langle H, R, s_{1}, n, C \rangle\)
  if \(H(R(x)) \neq \NULL\) and \(\scon\Sirx \notin C\)

    Similar to the above.
  
\item Case: \(\langle H, R, \IFNULL \Sirx \ \THEN s_{1} \ \ELSE
  \ s_{2}, n, C\rangle \rightarrow \langle H, R, s_{1}, n, C'
  \rangle\) if \(H(R(x)) = \NULL\), \(\scon\Sirx \notin C\) and
  \(\sassxn \notin C\), where \(C' = C \cup \{\sassx\}\)

  From assumption \( \Theta; \Gamma \vdash \langle H, R, \IFNULL\Sirx
  \ \THEN s_{1} \ \ELSE \ s_{2}, n, C \rangle : \langle P, F
  \rangle\), we have \(\Theta; \Gamma \vdash \IFNULL\Sirx \ \THEN
  s_{1} \ \ELSE \ s_{2} \COL P \) and \(OK_n(P, F)\). From the
  inversion of typing rules, we have \(\Theta; \Gamma \vdash s_{1}
  \COL P_1\), \(\Theta; \Gamma \vdash s_{2} \COL P_2\) and \((*x)(P_1,
  P_2) \le P\). By subtyping, \(\scon\Sirx \in C\) and \(\sassxn
  \notin C\) which are similar to \(\scon\Sirx \in F\) and \(\snnull
  \notin F\), we get \(\langle P, F \rangle \Longrightarrow \langle Q,
  F\cup \{\snull\} \rangle \) and \(P_1 \le Q\) for some \(Q\).
  
  We need to find \(P'\) and \(F'\) such that \(\Theta;\Gamma \vdash
  s_1 \COL P'\), \( \langle P, F \rangle \Longrightarrow \langle P',
  F' \rangle\) and \(\OK_n(P', F'\)). Take \(Q\) as \(P'\) and \(F\cup
  \{\snull\}\) as \(F'\). Then \(C' \thickapprox\ F'\), \( \langle P,F
  \rangle \rightarrow \langle P', F' \rangle\) and \(\OK_n(P', F')\)
  hold.  We also have \(\Theta; \Gamma \vdash s_1 \COL P'\) from
  \rn{T-Sub}, \(\Theta; \Gamma \vdash s_1 \COL P_1\) and \( P_1 \le
  Q\).

  %% We have \(\Theta; \Gamma \vdash \IFNULL \Sirx \ \THEN \ s_{1}
  %% \ \ELSE \ s_{2}\COL P\) and \(\OK_{n}(P, F)\). From the inversion of
  %% typing rules, we get \(\Theta; \Gamma \vdash s_1 : P_1\),
  %% \(\Theta; \Gamma \vdash s_2 : P_2\) and \( (*x)(P_1,P_2) \le
  %% P\). 

  %% \begin{itemize}
  %%  \item Subcase:  assume that \(\snnull \notin F\) and \(\scon\Sirx \in F \)

  %% By subtying, we get \( \langle P, F \rangle \rightarrow \langle Q,
  %% F' \rangle\) and \(P_1 \le Q\) for some \(Q\).

  %% We need to find \(P'\) s.t. \(\Theta;\Gamma\vdash s_1:P'\), \(
  %% \langle P, F \rangle \rightarrow \langle P', F' \rangle\) and \(OK_n(P',F')\).
  %% Take \(Q\) as \(P'\) and \( F \cup{\snull}\}\) as \(F'\), therefore
  %% \( \langle P, F \rangle \rightarrow \langle P', F' \rangle\) and
  %% \(OK_n(P',F')\) hold. From \(\Theta;\Gamma\vdash s_1:P_1\), \(P_1
  %% \le Q\) and \(\rn{T-Sub}\), \(\Theta;\Gamma\vdash s_1:P'\) holds.

  %% \item Subcase: assume that \(\scon\Sirx \notin F\)

  %% By subtying, we get \( \langle P, F \xLongrightarrow Q, F\) and
  %% \(P_1 \le Q\) for some \(Q\).

  %% We need to find \(P'\) s.t. \(\Theta;\Gamma\vdash s_1:P'\), \(
  %% \langle P, F \rangle \rightarrow \langle P', F' \rangle\) and \(OK_n(P',F')\).
  %% Take \(Q\) as \(P'\) and \(F\) as \(F'\), therefore
  %% \( \langle P, F \rangle \rightarrow \langle P', F' \rangle\) and
  %% \(OK_n(P',F')\) hold. From \(\Theta;\Gamma\vdash s_1:P_1\), \(P_1
  %% \le Q\) and \(\rn{T-Sub}\), \(\Theta;\Gamma\vdash s_1:P'\) holds.

  %% \item Subcase: assume that \(\scon\Sirx \notin F\)
  %% By subtying, we get \( \langle P, F \rangle \rightarrow \langle Q, F
  %% \rangle \) and \(P_1 \le Q\) for some \(Q\).

  %% We need to find \(P'\) s.t. \(\Theta;\Gamma\vdash s_1:P'\), \(
  %% \langle P, F \rangle \rightarrow \langle P', F' \rangle\) and \(OK_n(P',F')\).
  %% Take \(Q\) as \(P'\) and \(F\) as \(F'\), therefore
  %% \( \langle P, F \rangle \rightarrow  \langle P', F' \rangle\) and
  %% \(OK_n(P',F')\) hold. From \(\Theta;\Gamma\vdash s_1:P_1\), \(P_1
  %% \le Q\) and \(\rn{T-Sub}\), \(\Theta;\Gamma\vdash s_1:P'\) holds.
  %% \end{itemize}

  %% Therefore, this case holds.
  
  \item Case: \(\langle H, R, \IFNULL \Sirx \ \THEN s_{1} \ \ELSE
  \ s_{2}, n, C\rangle \rightarrow \langle H, R, s_{2}, n, C'
  \rangle\) if \(H(R(x)) \neq \NULL\), \(\scon\Sirx \notin C\) and
  \(\sassx \notin C\), where \(C' = C \cup \{\sassxn\}\)

  Similar to the above proof.
  
\end{itemize}
\end{pfof}  

%% \begin{corollary}
%% \label{cor:preservation}
%% If $OK_{n}(P, F)$, $\Theta; \Gamma \vdash s : P$, \(\vdash D \COL
%% \Theta\), and $\langle H,R,s,n\rangle \xlongrightarrow{\sigma} \langle
%% H',R',s', n' \rangle$, then there exists $P'$ such that (1) $ \Theta;
%% \Gamma \vdash s' : P'$, (2) \( \langle P, F \rangle
%% \xLongrightarrow{\sigma} \langle P', F' \rangle\), and (3)
%% \(OK_{n'}(P', F')\).
%% \end{corollary}

We write \(\langle H, R, s, n, C \rangle \xlongrightarrow{\rho}\) if
there is a transition \(\xlongrightarrow{\rho}\) from \(\langle H, R,
s, n, C \rangle\).

\begin{lemma}
\label{lem:enabled}
If \(\Theta; \Gamma \vdash \langle H, R, s, n, C \rangle \COL \langle
P,F \rangle\) and \(\langle H, R, s, n, C \rangle
\xLongrightarrow{\rho}\) and \(\rho \in \set{\Malloc, \Free}\), then
there exists \(P'\) and \(F'\) such that \( \langle P, F \rangle
\ \xLongrightarrow{\rho} \langle P', F'\rangle\).
\end{lemma}

\begin{proof}
Induction on the derivation of \(\Theta; \Gamma \vdash \langle H, R, s, n, C \rangle \COL \langle P, F \rangle\).
\end{proof}

\begin{pfof}{Lemma~\ref{lem:immediateSafety}}

By contradiction.  Assume \(\langle H, R, s, n, C \rangle
\xlongrightarrow{\rho} \OVERFLOW\). Then, \(n\) is \(0\) and \(\rho =
\Malloc\) from \rn{Sem-OutOfMem}.  From the assumption we have
\(\Theta;\Gamma \vdash s \COL P\) and \(\OK_0(P, F)\).  From
Lemma~\ref{lem:enabled}, there exists \(P'\) and \(F'\) such that \(
\langle P, F \rangle \xLongrightarrow{\Malloc} \langle P', F'
\rangle\).  However, this contradicts \(\OK_0(P, F)\).

\end{pfof}

\begin{pfof}{Theorem~\ref{thm1}}

We have \(\Theta;\emptyset \vdash s\COL P, \vdash D\COL \Theta\) and
\(\OK_n(P, F)\).

Suppose that there exists \(\sigma\) such that \(\langle \emptyset,
\emptyset, s, n, C\rangle \xlongrightarrow{\sigma} \langle H', R', s',
n', C'\rangle \xlongrightarrow{\rho} \OVERFLOW\).  Then, \(n' = 0\)
and \(\rho = \Malloc\).  From Lemma~\ref{lem:preservation}, there
exists \(P'\) and \(F'\) such that \(\Theta; \Gamma' \vdash s' \COL
P'\), \( \langle P, F \rangle \xLongrightarrow{\sigma} \langle P', F'
\rangle \), and \(\OK_0(P', F')\); hence \(\langle H', R', s',
0\rangle \xlongrightarrow{\Malloc}\).  However, this contradicts
Lemma~\ref{lem:immediateSafety}.

\end{pfof}
