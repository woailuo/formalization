\section{Type system}
\label{sec:typesystem}

\subsection{Types}
The syntax of the types is as follows.
\[
\begin{array}{rlcl}
  P & (\mbox{behavioral types})&::=& {\bf 0} \mid P_{1};P_{2} \mid \Free \mid \alpha \mid \mu\alpha.P \\
  &  &  & \mid \LET x = y \; \IN P \mid \LET x = \Malloc \; \IN P \\
   &  &  & \mid \LET x = \NULL \; \IN P \mid \LET x = *y \; \IN P \\
  &  &  & \mid (*x)(P_1,P_2) \mid \scon\Sirx P  \mid \Endconst\\
  \Gamma & (\mbox{variable type environment}) &::=& \set{x_1, x_2, \dots, x_n}\\
  \Psi & (\mbox{dependent function type}) &::=& (\vec{x})P\\
  \Theta & (\mbox{function type environment}) &::=& \set{f_1\COL \Psi_1,\dots,f_n\COL \Psi_n}\\
  k & (\mbox{constant values}) &::=& \snull \mid \snnull \mid \scon\Sirx   \\
  F & (\mbox{constant value environment}) &::=& \{k_1,...,k_n\} \\
\end{array}
\]

Behavioral types ranged over by \(P\) express the abstaction of
behaviors of a program. The type \({\bf 0}\) represents the “does
nothing” behavior; the type \(P_1;P_2\) describes a sequential
execution of behavioral type \(P_1\) and \(P_2\); The type \(\Malloc\)
expresses an allocation of a memory cell; the type \(\Free\)
represents a deallocation of a pointer; the type \(\mu \alpha.P\)
represents a recursive substitution of \(\alpha\) in \(P\) ; the type
\(\Sirx(P_1,P_2)\) represents that \(P_1\) or \(P_2\) is obtained
dependent on \(*x\), e.g., \(P_1\) is obtained if \(*x\) is not a null
pointer, otherwise \(P_2\); the type \(P_1 + P_2\) represents the
choice between \(P_1\) and \(P_2\); the \(\alpha\) is a type variable;
the type \(\scon\Sirx P\) represents that \(*x\) is a constant value
in type \(P\) ; the type \(\Endconst\) represents \(*x\) no longer be
a constant from this point.

A type environments for variables ranged over by \(\Gamma\) is a set
of variables without information about their types, because our
focus is the behavior of a program.

Dependent function types ranged over by \(\Psi\) represents the
behavior of a function. \(\vec{x}\) is the formal arguments of the
function, and the behavioral type \(P\) obtained dependent on
\(\vec{x}\).

Function types ranged over by \(\Theta\) is a mapping from function names to dependent function types.

\(k\) represents constant values information, where \(\snull\) represents
\(\Sirx\) is a null pointer; \(\snnull\) represents \(\Sirx\) is not a
null pointer; \(\scon\Sirx\) represents \(\Sirx\) should be a constant.

Constant value environment ranged over by \(F\) is a set of constant
values information.

\paragraph{Notation}
\(filter\_T(F, *x)\) is defined by a pseudcode as follows:
\[
\begin{array}{lcl}
  filter\_T(F, *x) &=& let \; F' = F - \scon\Sirx \; in \\
  & & if \; \scon\Sirx \notin \; F'\; then \; return \; (F' \backslash \{\snull,\snnull\})\\
  & & else \; return \; F'
\end{array}
\]


Figure~\ref{fig:bdRules} depicts semantics of behavioral types with
dependent types, and they are given by the labeled transition
system. The relation \( \langle P, F \rangle \xlongrightarrow{\rho}
\langle P', F' \rangle \) means that \(P\) can make an action
\(\rho\), and \(P\) turns into \(P'\) after it makes action \(\rho\);
\(F\) and \(F'\) record constant value environment before and after
making action \(\rho\) respectively.


\begin{figure}
 \begin{minipage}{\textwidth}


\infax[Tr-Skip]
{ \langle {\bf 0};P, F \rangle \rightarrow \langle P, F \rangle }

%% \begin{minipage}{0.5\textwidth}
%% \infax[Tr-Malloc]
%%  { \langle \Malloc, F \rangle \xlongrightarrow{\Malloc} \langle {\bf 0}, F \rangle }
%% \end{minipage}

\vspace{2mm}

\begin{minipage}{0.5\textwidth}
\infax[Tr-Free]
{ \langle \Free, F \rangle \xlongrightarrow{\Free} \langle {\bf 0}, F \rangle }
\end{minipage}
\begin{minipage}{0.5\textwidth}
\infax[Tr-Rec]
 { \langle \mu\alpha.P, F \rangle \rightarrow \langle
   [\mu\alpha.P/\alpha]P, F \rangle }
 \end{minipage}

\vspace{2mm}      

\begin{minipage}{0.5\textwidth}
\infax[Tr-ChoiceL]
{ \langle P_1 + P_2, F \rangle \rightarrow \langle P_1, F \rangle }
\end{minipage}
\begin{minipage}{0.5\textwidth}
\infax[Tr-ChoiceR]
{ \langle P_1 + P_2, F \rangle \rightarrow \langle P_2, F \rangle }
\end{minipage}

\infrule[Tr-Seq]
{ \langle P_1, F \rangle \xlongrightarrow{\rho} \langle P_1', F' \rangle }
{ \langle P_1;P_2, F \rangle \xlongrightarrow{\rho} \langle P_1';P_2, F' \rangle }

\infax[Tr-LetMalloc]
{ \langle \LET x = \Malloc \; \IN P, F \rangle \xlongrightarrow{\Malloc(x')} \langle [x'/x]P, F \rangle }

\infax[Tr-LetXY]
{ \langle \LET x = y \; \IN P, F \rangle \rightarrow \langle [x'/x]P, F \rangle }

\infax[Tr-LetX*Y]
{ \langle \LET x = *y \; \IN P, F \rangle \rightarrow \langle [x'/x]P, F \rangle }

\infax[Tr-LetX]
{ \langle \LET x = \NULL \; \IN P, F \rangle \rightarrow \langle [x'/x]P, F \rangle }


\infax[Tr-Const]
{ \langle \scon\Sirx P, F \rangle \rightarrow \langle P;\Endconst, F\cup \{\scon\Sirx\} \rangle }

\infrule[Tr-Endconst]
{F' = filter\_T(F, *x)}
{ \langle \Endconst, F \rangle \rightarrow \langle {\bf 0}, F' \rangle }

%% \infrule[Tr-NNullNotIn]
%% { \snnull \notin F \andalso \scon\Sirx \in F}
%% { \langle \Sirx (P_1,P_2), F \rangle \rightarrow \langle P_1, F\cup\{\snull\} \rangle }

\begin{minipage}{0.5\textwidth}
\infrule[Tr-NotConst1]
{ \scon\Sirx \notin F }
{ \langle \Sirx (P_1,P_2), F \rangle \xlongrightarrow{\snull} \langle P_1, F \rangle }
\end{minipage}
\begin{minipage}{0.5\textwidth}
\infrule[Tr-NotConst2]
{ \scon\Sirx \notin F }
{ \langle \Sirx (P_1,P_2), F \rangle \xlongrightarrow{\snnull} \langle P_2, F \rangle }
\end{minipage}
\vspace{2mm}
%% \infrule[Tr-NullNotIn]
%% { \snull \notin F \andalso \scon\Sirx \in F }
%% { \langle \Sirx (P_1,P_2), F \rangle \rightarrow \langle P_2, F\cup\{\snnull\} \rangle }

\begin{minipage}{0.5\textwidth}
\infrule[Tr-NullIn]
{ \snull \in F \andalso \scon\Sirx \in F }
{ \langle \Sirx (P_1,P_2), F \rangle \rightarrow \langle P_1, F \rangle }
\end{minipage}
\begin{minipage}{0.5\textwidth}
\infrule[Tr-NNullIn]
{ \snnull \in F \andalso \scon\Sirx \in F }
{ \langle \Sirx (P_1,P_2), F \rangle \rightarrow \langle P_2, F \rangle }
\end{minipage}


\infrule[Tr-NNullNotIn1]
{ \snull, \snnull \notin F \andalso \scon\Sirx \in F }
{ \langle \Sirx (P_1,P_2), F \rangle \xlongrightarrow{\snull} \langle P_1, F\cup{\snull} \rangle }

\infrule[Tr-NNullNotIn2]
{ \snull, \snnull \notin F \andalso \scon\Sirx \in F }
{ \langle \Sirx (P_1,P_2), F \rangle \xlongrightarrow{\snnull} \langle P_2, F\cup{\snnull} \rangle }

 
%%   \label{df:fv}
%%   \mbox{The set of free variables of behavioral type \(P\) is defined as
%%     follows:}

%%   \( \mathbf{FV}(\bf{0}) = \mathbf{FV}(\Malloc) =
%%   \mathbf{FV}(\Free) = \mathbf{FV}(\alpha) = \emptyset \) 

%%   \( \mathbf{FV}(P_1 + P_2) = \mathbf{FV}(P_1;P_2) = \mathbf{FV}(P_1) \cup \mathbf{FV}(P_2)\)

%%   \( \mathbf{FV}((x)(P_1,P_2)) \) = \{x\} \(\cup \ \mathbf{FV}(P_1) \cup
%%   \mathbf{FV}(P_2)\)
%% \end{myDef}
\end{minipage}
\caption{semantics of behavioral types with dependent types.}
\label{fig:bdRules}
\end{figure}

\subsection{Typing rules}
The type judgment for statements is of the form \(\Theta; \Gamma
\vdash s : P \), which represents that under the function type
environment \(\Theta\) and the variable type environment \(\Gamma\),
the abstracted behavioral type of statement \(s\) is \(P\).

Before showing typing rules for statements in
Figure~\ref{fig:TypingRules}, we need explain several important
definitions. The first one is \(OK_n(P, F)\), a predicate, where \(P\)
represents the behavior of a program which consumes at most \(n\)
memory cells under constant value environment \(F\).

\begin{myDef}[\(\sharp_{\rho}(\sigma)\)]
\label{df:sharf}
\(\sharp_{\rho}(\sigma)\) is the number of \(\rho\) in the sequence
\(\sigma\).
\end{myDef}

%% \begin{myDef}
%% \label{df:const}
%% \(\sconst\) means:\\
%% \( \forall \sigma',\sigma''\). \(\sigma'\) is a subsequence
%% of \(\sigma\), and \(\sigma' = \Startconst;\sigma'';\Endconst\) or \(\sigma' = \sigma'';\Endconst\). The \(\sigma''\) does not contain \(\Startconst\) or \(\Endconst \), and it also does not contain both \(\sassx\) and \(\sassxn\).
%% \end{myDef}


\begin{myDef}
\label{df:okn}
\(OK_{n}(P, F)\) holds if, (1) \( \forall P'\) and \(\sigma\). if \( \langle P, F \rangle
\xlongrightarrow{\sigma} \langle P', F' \rangle \), then \( \sharp_m(\sigma) - \sharp_f(\sigma) \le n\) and (2) \( OK(F) \)
\end{myDef}

\begin{myDef}
\label{df:okf}
\(OK(F)\) holds if F does not contain both \( \snull \) and \( \snnull \).
\end{myDef}

Intuitively, \(OK_n(P, F)\) represents at very running steps, the
number of memory cells a program consumed will not exceed the number
of memory cells the program requires.

\begin{myDef}[Subtyping]
\( F\vdash P_1 \le P_2\) is the largest relation such that, for any
\(P_1'\), \(F'\) and \(\rho\), if \( \langle P_1, F \rangle
\xlongrightarrow{\rho} \langle P_1', F' \rangle \), then there exists
\(P_2'\) such that \( \langle P_2, F \rangle \xLongrightarrow{\rho}
\langle P_2', F' \rangle \) and \( F'\vdash P_1' \le P_2'\).  We write
\( P_1 \le P_2\) if \(F\vdash P_1 \le P_2\) for any F.
\label{df:subtype}
\end{myDef}

\begin{figure}
\begin{minipage}{\textwidth}

\begin{minipage}{0.5\textwidth}
\infax[T-Skip]
{\Theta ; \Gamma \vdash \SKIP : \mathbf{0}}
\end{minipage}
\begin{minipage}{0.5\textwidth}
\infrule[T-Seq]
{\Theta ; \Gamma \vdash s_{1} : P_{1} \Rtab \Theta ; \Gamma \vdash s_{2} : P_{2}}
{\Theta ; \Gamma \vdash s_{1} ; s_{2} : P_{1};P_{2} }
\end{minipage}

\vspace{2mm}

\begin{minipage}{0.5\textwidth}
\infax[T-Assign]
{\Theta ; \Gamma, x, y \vdash *x \leftarrow y : \mathbf{0} }
\end{minipage}
\begin{minipage}{0.5\textwidth}
\infax[T-Free]
{\Theta ; \Gamma, x \vdash \Free(x) : \Free}
\end{minipage}

\vspace{2mm}

\begin{minipage}{0.5\textwidth}
\infrule[T-Malloc]
{\Theta ; \Gamma,x \vdash s : P  \Rtab}
{\Theta ; \Gamma \vdash \LET x = \MALLOC \; \IN s :\LET x = \Malloc \; \IN P}
\end{minipage}
\begin{minipage}{0.5\textwidth}
\infrule[T-LetEq]
{\Theta ; \Gamma , x, y  \vdash s : P}
{\Theta ; \Gamma, y \vdash \LET x = y \; \IN s : \LET x = y \; \IN P}
\end{minipage}

\vspace{2mm}

\begin{minipage}{0.5\textwidth}
\infrule[T-LetDeref]
{\Theta ; \Gamma , x, y  \vdash s : P  \Rtab}
{\Theta ; \Gamma, y \vdash \LET x = *y \; \IN s : \LET x = *y \; \IN P}
\end{minipage}
\begin{minipage}{0.5\textwidth}
\infrule[T-LetNull]
{\Theta ; \Gamma, x  \vdash s : P  \Rtab}
{\Theta ; \Gamma \vdash \LET x = \NULL \; \IN s : \LET x = \NULL \; \IN P}
\end{minipage}

\infax[T-Endconst] 
{\Theta ; \Gamma,x \vdash \Endconst : \Endconst}

\infrule[T-Const] 
{\Theta ; \Gamma, x \vdash s : P }
{\Theta ; \Gamma,x \vdash \scon\Sirx s : \scon\Sirx P}


%%\begin{minipage}{0.5\textwidth}  \andalso x \in \mathbf{FV}(P_1) \cup \mathbf{FV}(P_2)
\infrule[T-IfNull] 
{\Theta ; \Gamma, x \vdash s_{1} : P_1 \andalso \Theta ; \Gamma, x \vdash s_{2} : P_2}
{\Theta ; \Gamma, x \vdash \IFNULL\Sirx \; \THEN s_{1}\; \ELSE s_{2} : (*x)(P_1,P_2)}
%%\end{minipage}
%%\begin{minipage}{0.5\textwidth}
\infax[T-Call] {\Theta, f \COL (\vec{y})P; \Gamma, \vec{x} \vdash f(\vec{x}) :  P[\vec{x}/\vec{y}]}
%%\end{minipage}

%%\vspace{2mm} 

\infrule[T-Sub]
{\Theta ; \Gamma \vdash s : P_{1} \andalso P_{1} \le P_{2}}
{\Theta ; \Gamma \vdash s : P_{2}}

\infrule[T-Def]
        { \Theta(f) = (\vec{x})P \andalso \DOM(D) = \DOM(\Theta)
          \andalso \Theta; x_1,\dots,x_n \vdash s \COL P
          \mbox{ for each \(f \mapsto (x_1,\dots,x_n)s \in D\)}
        }
        {\vdash D \COL \Theta}
% Program
\infrule[T-Program]
{\vdash D : \Theta \andalso \Theta; \emptyset\vdash s : P \Rtab OK_{n}(P, F)}
{\vdash \langle D, s \rangle : n}

\end{minipage}
\caption{typing rules}
\label{fig:TypingRules}
\end{figure}

Figure~\ref{fig:TypingRules} shows the typing rules. For example, the
rule \rn{T-IfNull} represents the behavior of \(\IFNULL\Sirx \; \THEN
s_{1}\; \ELSE s_{2}\) is abstracted as \(\Sirx(P_1, P_2)\) where
\(P_1\) and \(P_2\) are the behavior of \(s_1\) and \(s_2\)
respectively; this conditional statement means that executing \(s_1\)
if \(\Sirx\) is a null pointer, otherwise \(s_2\).  The typing rule
\rn{T-Program} represents a program requires at most \(n\) memory
cells during running under the predication \(OK_n(P, F)\), where \(P\)
is behavioral type of statement \(s\).


\subsection{Type soundness}
\begin{theorem}\label{thm1}
If $\vdash \langle D, s \rangle : n$ for some \(n\), then \(\langle D,
s \rangle\) is totally memory-leak free.
\end{theorem}

The proof is based on the following lemmas: preservation and lack of
immediate overflow.

%% \begin{myDef}
%%   \label{df:Rab}
%%   Rab(\(\langle P, F \rangle \xlongrightarrow{\rho} \langle P', F' \rangle\)) should be
%%   \[
%%     \left\{
%%    \begin{array}{ll}
%%      \langle P, F \rangle \xlongrightarrow{\rho} \langle P', F' \rangle & \rho = \{\Free, \Malloc\}\\
%%      \langle P, F \rangle \rightarrow \langle P', F' \rangle & \mbox{Otherwise.}
%%    \end{array}
%%    \right.
%% \]
%% \end{myDef}

\begin{myDef}
  \label{df:Rdu}
 we write \( \Theta; \Gamma \vdash \langle H, R, s, n, C \rangle : \langle P,
  F \rangle\), if \(\Theta; \Gamma \vdash s : P\) and \(OK_n(P, F)\) with \(C \thickapprox F\).  
\end{myDef}

%% \begin{lemma}[Preservation]
%% \label{lem:preservation}
%% If \(OK_{n}(P, F)\), \(\Theta; \Gamma \vdash s : P \), \(\vdash D \COL
%% \Theta\), and \( \langle H, R, s, n, C \rangle \xlongrightarrow{\rho}
%% \langle H, R', s', n', C' \rangle \), then there exists \(P'\) and \(F'\)
%% such that (1) \( \Theta; \Gamma^{'} \vdash s' : P'\), (2)
%% Rab(\(\langle P, F \rangle \xlongrightarrow{\rho} \langle P', F'
%% \rangle\)), and (3) \(OK_{n'}(P', F')\).
%% \end{lemma}
 
\begin{lemma}[Preservation]
\label{lem:preservation}
suppose that \( \Theta; \Gamma \vdash \langle H, R, s, n, C \rangle :
\langle P, F \rangle\). If \( \langle H, R, s, n, C \rangle
\xlongrightarrow{\rho} \langle H', R', s', n', C' \rangle\) then
\(\exists P', F'\) s.t. (1) \( \Theta; \Gamma \vdash \langle H', R',
s', n', C' \rangle : \langle P', F' \rangle\) and (2) \(\langle P, F
\rangle \xLongrightarrow{\rho} \langle P', F'\rangle \).
\end{lemma}


%% \begin{lemma}[Lack of immediate overflow]
%% \label{lem:immediateSafety}
%% If \(\Theta; \Gamma \vdash s \COL P \), \(\vdash D \COL \Theta\), and
%% \(\OK_n(P)\), then \(\langle H, R, s, n \rangle
%% \not\xlongrightarrow{\Malloc} \OVERFLOW \).
%% \end{lemma}

\begin{lemma}[Lack of immediate overflow]
\label{lem:immediateSafety}
If \( \Theta; \Gamma \vdash \langle H, R, s, n, C \rangle :
\langle P, F \rangle\), then \(\langle H, R, s, n, C \rangle
\not\xlongrightarrow{\Malloc} \OVERFLOW \).
\end{lemma}
