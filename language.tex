
\section{Language \(\mathcal{L}\)}\label{sec:language}

\begin{eqnarray*}
  x,y,z,\dots \mbox{ (variables)} &\in& \VAR\\
  s \mbox{ (statements)} & ::= &  \SKIP \mid s_{1};s_{2} \mid *x \leftarrow y \mid \Free(x) \\
  & \mid & \LET x = \MALLOC \IN s \mid \LET x = \NULL\ \IN s  \\
  & \mid & \LET x = y \; \IN s \mid   \LET x = *y \; \IN s \\
  & \mid & \IFNULL(*x) \; \THEN s_{1}\; \ELSE s_{2} \mid f(\vec{x})\\
  & \mid & \scon\Sirx s \mid \Endconst  \\
  d \mbox{ (proc. defs.)} & ::= & \set{f \mapsto (x_1,\dots,x_n)s}\\
  D \mbox{ (definitions) } &::=& \langle d_1 \cup \dots \cup d_n \rangle\\
  P \mbox{ (programs)} &::=& \langle D, s \rangle\\
  E \mbox{ (context)} &::=& E;s \mid \scon\Sirx E \mid [ ] 
\end{eqnarray*}

\subsection{Operational semantics}
\label{sec:languageSemantics}
\begin{itemize}
\item \(H\), a \emph{heap}, is a finite mapping from \(\mathcal{H}\)
  to \(\mathcal{H} \cup \set{\NULL}\);
\item \(R\), an \emph{environment}, is a finite mapping from \(\VAR\)
  to \(\mathcal{H} \cup \set{\NULL}\);
\item \(s\) is the statement that is being executed; and
\item \(n\) is a natural number that represents the number of memory
  cells available for allocation.
\item \(\rho   \mbox{ (label)} ::= \Malloc \mid \Free \mid \sassx \mid \sassxn\)   \\
 \( \Rtab \mid \Startconst \mid \Endconst \)  \\
\end{itemize}

The operational semantics is given by relation \(\langle H, R, s, n, C
\rangle \xlongrightarrow{\rho}_D \langle H', R', s', n', C \rangle\)
where \(\rho\), an \emph{action}, is \(\Malloc\), \(\Free\),
\(\sassx\), \(\sassxn\),, \(\Startconst\), \(\Endconst \) or \(\tau\).
The action \(\Malloc\) expresses an allocation of a memory cell;
\(\Free\) expresses a deallocation; \(\sassx\) and \(\sassxn\) express
the guard part of conditional are \( \xnull \) and \(\xnnull\)
respectively; \(\Startconst\) means \(*x\) should be constant from
here; \(\Endconst\) means the \(*x\) will not be guaranteed a
constant from here; \(\tau\) expresses the other actions.  We often omit
\(\tau\) in \(\xlongrightarrow{\tau}_D\).  We use a metavariable
\(\sigma\) for a finite sequence of actions \(\rho_1\dots\rho_n\).  We
write \(\xlongrightarrow{\rho_1\dots\rho_n}_D\) for
\(\xlongrightarrow{\rho_1}_D\xlongrightarrow{\rho_2}_D\dots\xlongrightarrow{\rho_n}_D\).
We write \(\xLongrightarrow{\rho}_D\) for
\(\xlongrightarrow{}_D^*\xlongrightarrow{\rho}_D\xlongrightarrow{}_D^*\).
We write \(\xLongrightarrow{\rho_1\dots\rho_n}_D\) for
\(\xLongrightarrow{\rho_1}_D\dots\xLongrightarrow{\rho_n}_D\).

%% \begin{figure}
\begin{minipage}{\textwidth}

\infax[Sem-ConstSkip]
{\langle H, R, \Endconst, n, C \uplus {x} \rangle \xlongrightarrow{\Endconst}_{D} \langle H, R, \SKIP , n, C  \rangle}

\vspace{2mm}

\infax[Sem-ConstSeq]
{\langle H, R, \scon\Sirx s, n, C \rangle
\xlongrightarrow{\Startconst}_{D}
\langle H, R, s;\Endconst , n, C \cup {x} \rangle}

\vspace{2mm}  

\infax[Sem-Skip]
{\langle H, R, \SKIP;s, n, C \rangle
\longrightarrow_{D}
\langle H, R, s, n, C\rangle}

\vspace{2mm}

\infrule[Sem-Seq]
{\langle H, R, s_1, n, C \rangle \xlongrightarrow{\rho}_{D} \langle H', R', s_1', n', C \rangle}
{\langle H, R, s_1;s_2, n, C \rangle \xlongrightarrow{\rho}_{D} \langle H', R', s_1';s_2, n', C \rangle}
 
\vspace{2mm}

\infrule[Sem-LetNull]
{x' \notin \DOM(R)}
{\langle H\coma R\coma  \LET x = \NULL \ \IN s , n, C \rangle
  \longrightarrow_{D}
  \langle H\coma R\Lfc x' \mapsto \NULL \Rfc \coma   \Lb x'/x \Rb s , n, C  \rangle }

\vspace{2mm}

\infrule[Sem-LetEq]
{x' \notin \DOM(R)}
{\langle H\coma R\coma \LET x = y \; \IN s , n, C \rangle
  \longrightarrow_{D}
  \langle H\coma R\Lfc x' \mapsto R(y) \Rfc \coma   \Lb x'/x \Rb s , n, C\rangle }

\vspace{2mm}

\infrule[Sem-IfNullT]
{H(R(x)) = \NULL}
{\langle H \coma R \coma \IFNULL\Sirx \ \THEN   s_{1}\ \ELSE\  s_{2} \coma  n, C \rangle
  \xlongrightarrow{\sassx}_{D}
  \langle H\coma R\coma s_{1} \coma n, C\rangle}

\vspace{2mm}

\infrule[Sem-IfNullF]
{H(R(x)) \neq \NULL}
{\langle H \coma R \coma \IFNULL\Sirx\ \THEN  s_{1}\ \ELSE  s_{2} \coma  n, C\rangle
  \xlongrightarrow{\sassxn}_{D}
  \langle H\coma R\coma s_{2} \coma  n, C\rangle}

\vspace{2mm}

\infrule[Sem-Call]
{D(f) = (\vec{y})s}
{ \langle H\coma R\coma  f(\vec{x}) , n, C\rangle
  \longrightarrow_{D}
  \langle H\coma R\coma  \Lb \vec{x}/\vec{y} \Rb s , n, C\rangle}

\vspace{2mm}

\infax[Sem-Assign]
{ \langle H \set{R(x) \mapsto v}, R, *x \leftarrow y , n, C\rangle \xlongrightarrow{}_{D}
  \langle H \Lfc R(x) \mapsto R(y) \Rfc , R, \SKIP , n, C\rangle }

\vspace{2mm}

\infrule[Sem-LetDeref]
{x' \notin \DOM(R) \andalso R(y) \in \DOM(H)}
{\langle H\coma R\coma  \LET x = *y \; \IN s , n, C\rangle
  \longrightarrow_{D}
  \langle H\coma R\Lfc x' \mapsto H(R(y)) \Rfc \coma   \Lb x'/x \Rb s , n, C\rangle }

\vspace{2mm}

\infrule[Sem-Free]
{R(x) \neq \NULL \mbox{ and } R(x) \in \DOM(H)}
{\langle H\set{R(x) \mapsto v}\coma R\coma \Free(x) , n, C \rangle \xlongrightarrow{\Free}_{D}
  \langle H\backslash R(x) \coma R \coma \SKIP , n+1, C \rangle}

\vspace{2mm}

\infrule[Sem-Malloc]
{l \notin \DOM(H) \andalso n > 0}
{\langle H\coma R\coma  \LET x = \Malloc() \; \IN s , n, C\rangle
  \xlongrightarrow{\Malloc}_{D}
  \langle H \Lfc l \mapsto v\Rfc \coma R\Lfc x' \mapsto l \Rfc \coma   \Lb x'/x \Rb s , n-1, C  \rangle }

\vspace{2mm}

\begin{minipage}{0.5\textwidth}
\infrule[Sem-AssignExn]
{R(x) = \NULL \mbox{ or } R(x) \notin \DOM(H)}
{\langle H\coma R\coma  *x \leftarrow y , n, C\rangle
  \longrightarrow_{D} \MEMEX }
\end{minipage}
\begin{minipage}{0.5\textwidth}
\infrule[Sem-DerefExn]
{R(y) = \NULL \mbox{ or } R(y) \notin \DOM(H)}
{\langle H\coma R\coma  \LET x = *y \; \IN s, n, C\rangle
    \longrightarrow_{D} \MEMEX}
\end{minipage}

\infrule[Sem-FreeExn]
{R(x) = \NULL \mbox{ or } R(x) \notin \DOM(H)}
{\langle H\coma R\coma \Free(x) , n, C\rangle \xlongrightarrow{\Free}_{D} \MEMEX}

\infrule[Sem-ConstExn]
        { H(R(x)) \neq H'(R(x))
          \Rtab
          \langle H, R, s, n, C \rangle \xlongrightarrow{\rho}_{D} \langle H', R', s', n', C \rangle}
{\langle H, R, \scon\Sirx s, n, C \rangle \xlongrightarrow{\rho}_{D} \CONSTEX }


\infax[Sem-OutOfMem]
{ \langle H\coma R\coma \LET x = \Malloc() \ \IN s ,  0, C  \rangle    \xlongrightarrow{\Malloc}_{D}
  \OVERFLOW}

\end{minipage}

%% \caption{Operational semantics of \(\mathcal{L}\).}
%% \label{fig:transitionRules}
%% \end{figure}
