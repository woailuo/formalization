
\section{Language \(\mathcal{L}\)}\label{sec:language}
In this section, we provide an imperative language \(\mathcal{L}\) with memory allocation and deallocation primitives, and for simplification we only use pointers as values.

The syntax of language \(\mathcal{L}\) is as follows.

\begin{eqnarray*}
  x,y,z,\dots \mbox{ (variables)} &\in& \VAR\\
  s \mbox{ (statements)} & ::= &  \SKIP \mid s_{1};s_{2} \mid *x \leftarrow y \mid \Free(x) \\
  & \mid & \LET x = \MALLOC \IN s \mid \LET x = \NULL\ \IN s  \\
  & \mid & \LET x = y \; \IN s \mid   \LET x = *y \; \IN s \\
  & \mid & \IFNULL(*x) \; \THEN s_{1}\; \ELSE s_{2} \mid f(\vec{x})\\
  & \mid & \scon\Sirx s \mid \Endconst  \\
  d \mbox{ (proc. defs.)} & ::= & \set{f \mapsto (x_1,\dots,x_n)s}\\
  D \mbox{ (definitions) } &::=& \langle d_1 \cup \dots \cup d_n \rangle\\
  P \mbox{ (programs)} &::=& \langle D, s \rangle\\
  E \mbox{ (context)} &::=& E;s \mid \scon\Sirx E \mid [ ] 
\end{eqnarray*}

The \(\VAR\) is a countably infinite set of \emph{variables} and every
variable is a pointer; the statement \(\SKIP\) means "does nothing";
the statement \(s_1;s_2\) is a sequential execution of \(s_1\) and
\(s_2\); the statement \(*x \leftarrow y\) updates the content of cell
which is pointed to by \(x\) with the value \(y\); the statement
\(\Free(x)\) deallocates a memory cell which is pointed to by pointer
\(x\); the statement \(\LET x = e\ \IN s\) evaluates the expression
\(e\), binds \(x\) to the result, and executes \(s\); the expression
\(\Malloc()\) allocates a new memory cell; the expression \(\NULL\)
evaluates to the null pointer; the expression \(*y\) means
dereferencing a memory cell pointed by \(y\); the statement
\(\IFNULL(*x)\ \THEN\ s_1\ \ELSE\ s_2\) executes \(s_1\) if \(*x\) is
\(\NULL\) and executes \(s_2\) otherwise; the statement \(f(\vec{x})\)
expresses a procedure \(f\) with arguments \(\vec{x}\); the statement
\(\scon\Sirx s\) means \(\Sirx\) is a constant in statement \(s\); the
statement \(\Endconst\) means from this point \(\Sirx\) maybe not
constant.

The \(d\) represents a procedure definition which maps a procedure
name \(f\) to its procedure body \((\vec{x})s\); The \(D\) represents
a set of procedure definitions \(\langle d_1 \cup\dots d_n \rangle\),
and each definition is distinct; The pair \(\langle D, s \rangle \)
represents a program, where \(D\) is a set of definitions and \(s\) is
a main statement; the \(E\) represents evaluation environment.

\subsection{Operational semantics}
\label{sec:languageSemantics}
In this section we introduce operational semantics of language
\(\mathcal{L}\). We assume there is a countable infinite set
\(\mathcal{H}\) of \(\emph{heap addresses}\) ranged over by \(l\).

We use a quadruple configuration \(\langle H, R, s, n \rangle\) to
express a run-time state. Each elements in the configuration is as
follows.

\begin{itemize}
\item \(H\), a \emph{heap}, is a finite mapping from \(\mathcal{H}\)
  to \(\mathcal{H} \cup \set{\NULL}\);
\item \(R\), an \emph{environment}, is a finite mapping from \(\VAR\)
  to \(\mathcal{H} \cup \set{\NULL}\);
\item \(s\) is the statement that is being executed; 
\item \(n\) is a natural number that represents the number of memory
  cells available for allocation.
%% \item \(C\) is a set of variables
\end{itemize}

The operational semantics of language \(\mathcal{L}\) is given by
relation \(\langle H, R, s, n \rangle \xlongrightarrow{\rho}_D
\langle H', R', s', n' \rangle\). The label \(\rho\) is as follows.

\begin{eqnarray*}
 \rho \mbox{ (label)} &::=& \Malloc \mid \Free \mid \sassx \mid \sassxn  \\
 &&\mid \Startconst \mid \Endconst \mid \tau  \\
\end{eqnarray*}

The \(\rho\), an \emph{action}, is \(\Malloc\), \(\Free\), \(\sassx\),
\(\sassxn\), \(\Startconst\), \(\Endconst \) or \(\tau\).  The action
\(\Malloc\) expresses an allocation of a memory cell; \(\Free\)
expresses a deallocation of a memory cell; \(\sassx\) and \(\sassxn\)
express the guard part of conditional are \( \xnull \) and \(\xnnull\)
respectively; \(\Startconst\) means \(*x\) should be constant from
this point; \(\Endconst\) means the \(*x\) no longer be a constant
from this point; \(\tau\) expresses the other actions.  We often omit
\(\tau\) in \(\xlongrightarrow{\tau}_D\).  We use a metavariable
\(\sigma\) for a finite sequence of actions \(\rho_1\dots\rho_n\).  We
write \(\xlongrightarrow{\rho_1\dots\rho_n}_D\) for
\(\xlongrightarrow{\rho_1}_D\xlongrightarrow{\rho_2}_D\dots\xlongrightarrow{\rho_n}_D\).
We write \(\xLongrightarrow{\rho}_D\) for
\(\xlongrightarrow{}_D^*\xlongrightarrow{\rho}_D\xlongrightarrow{}_D^*\).
We write \(\xLongrightarrow{\rho_1\dots\rho_n}_D\) for
\(\xLongrightarrow{\rho_1}_D\dots\xLongrightarrow{\rho_n}_D\).

Figure~\ref{fig:transitionRules} represents the relation \(\xlongrightarrow{\rho}_D\).

\begin{itemize}
\item \rn{Sem-ConstSkip}: That a memory cell pointed to by \(x\) no
  longer be a constant is expressed by doing nothing.
\item \rn{Sem-ConstSeq}: That a memory cell pointed to by\(x\) should
  be a constant in a stamtement \(s\) is expressed by adding a statement \(\Endconst\) at the end of statement \(s\).
\item \rn{Sem-Free}: Deallocation of a memory cell pointed to by \(x\)
  is expressed by deleting the entry for \(R(x)\) from the heap.  This
  action increments the number of available cells (i.e., \(n\)) by one
  (i.e., \(n+1\)).
\item \rn{Sem-Malloc} and \rn{Sem-OutOfMem}: Allocation of a memory
  cell is expressed by adding a fresh entry to the heap.  This action is
  allowed only if the number of available cells is positive; if the
  number is zero, then the configuration leads to an error state
  \(\OVERFLOW\).
\item \rn{Sem-AssignExn},\rn{Sem-FreeExn},\rn{Sem-DerefExn} and
  \rn{Sem-FreeExn} : These rules express an illegal access to memory.
  If such action is performed, then the configuration leads to
  exceptional state \(\MEMEX\).  This state \(\MEMEX\) is not seen as
  an erroneous state in the current paper, hence is not excluded by
  the type system in Section~\ref{sec:typesystem}.  The command
  \(\FREE(x)\), if \(x\) is a null pointer, leads to \(\MEMEX\) in the
  current semantics, although it is equivalent to \(\SKIP\) in the C
  language.
\item \rn{Sem-ConstExn}: this rule express that if a constant \(*x\)
  is changed in statement \(s\) it will raise \(\CONSTEX\) exception.

  
\end{itemize}

\begin{figure}

  \begin{minipage}{\textwidth}

  \infax[Sem-ConstSkip]
        {\langle H, R, \Endconst, n
  \rangle \xlongrightarrow{\Endconst}_{D} \langle H, R, \SKIP , n
  \rangle}

\vspace{2mm}

\infax[Sem-ConstSeq]
{\langle H, R, \scon\Sirx s, n \rangle
\xlongrightarrow{\Startconst}_{D}
\langle H, R, s;\Endconst , n \rangle}

\vspace{2mm}  

\infax[Sem-Skip]
{\langle H, R, \SKIP;s, n \rangle
\longrightarrow_{D}
\langle H, R, s, n \rangle}

\vspace{2mm}

\infrule[Sem-Seq]
{\langle H, R, s_1, n  \rangle \xlongrightarrow{\rho}_{D} \langle H', R', s_1', n' \rangle}
{\langle H, R, s_1;s_2, n \rangle \xlongrightarrow{\rho}_{D} \langle H', R', s_1';s_2, n' \rangle}
 
\vspace{2mm}

\infrule[Sem-LetNull]
{x' \notin \DOM(R)}
{\langle H\coma R\coma  \LET x = \NULL \ \IN s , n \rangle
  \longrightarrow_{D}
  \langle H\coma R\Lfc x' \mapsto \NULL \Rfc \coma   \Lb x'/x \Rb s , n  \rangle }

\vspace{2mm}

\infrule[Sem-LetEq]
{x' \notin \DOM(R)}
{\langle H\coma R\coma \LET x = y \; \IN s , n  \rangle
  \longrightarrow_{D}
  \langle H\coma R\Lfc x' \mapsto R(y) \Rfc \coma   \Lb x'/x \Rb s , n\rangle }

\vspace{2mm}

\infrule[Sem-IfNullT]
{H(R(x)) = \NULL}
{\langle H \coma R \coma \IFNULL\Sirx \ \THEN   s_{1}\ \ELSE\  s_{2} \coma  n \rangle
  \xlongrightarrow{\sassx}_{D}
  \langle H\coma R\coma s_{1} \coma n \rangle}

\vspace{2mm}

\infrule[Sem-IfNullF]
{H(R(x)) \neq \NULL}
{\langle H \coma R \coma \IFNULL\Sirx\ \THEN  s_{1}\ \ELSE  s_{2} \coma  n\rangle
  \xlongrightarrow{\sassxn}_{D}
  \langle H\coma R\coma s_{2} \coma  n\rangle}

\vspace{2mm}

\infrule[Sem-Call]
{D(f) = (\vec{y})s}
{ \langle H\coma R\coma  f(\vec{x}) , n\rangle
  \longrightarrow_{D}
  \langle H\coma R\coma  \Lb \vec{x}/\vec{y} \Rb s , n \rangle}

\vspace{2mm}

\infax[Sem-Assign]
{ \langle H \set{R(x) \mapsto v}, R, *x \leftarrow y , n\rangle \xlongrightarrow{}_{D}
  \langle H \Lfc R(x) \mapsto R(y) \Rfc , R, \SKIP , n \rangle }

\vspace{2mm}

\infrule[Sem-LetDeref]
{x' \notin \DOM(R) \andalso R(y) \in \DOM(H)}
{\langle H\coma R\coma  \LET x = *y \; \IN s , n \rangle
  \longrightarrow_{D}
  \langle H\coma R\Lfc x' \mapsto H(R(y)) \Rfc \coma   \Lb x'/x \Rb s , n\rangle }

\vspace{2mm}

\infrule[Sem-Free]
{R(x) \neq \NULL \mbox{ and } R(x) \in \DOM(H)}
{\langle H\set{R(x) \mapsto v}\coma R\coma \Free(x) , n \rangle \xlongrightarrow{\Free}_{D}
  \langle H\backslash R(x) \coma R \coma \SKIP , n+1 \rangle}

\vspace{2mm}

\infrule[Sem-Malloc]
{l \notin \DOM(H) \andalso n > 0}
{\langle H\coma R\coma  \LET x = \Malloc() \; \IN s , n\rangle
  \xlongrightarrow{\Malloc}_{D}
  \langle H \Lfc l \mapsto v\Rfc \coma R\Lfc x' \mapsto l \Rfc \coma   \Lb x'/x \Rb s , n-1  \rangle }

\vspace{2mm}

\begin{minipage}{0.5\textwidth}
\infrule[Sem-AssignExn]
{R(x) = \NULL \mbox{ or } R(x) \notin \DOM(H)}
{\langle H\coma R\coma  *x \leftarrow y , n\rangle
  \longrightarrow_{D} \MEMEX }
\end{minipage}
\begin{minipage}{0.5\textwidth}
\infrule[Sem-DerefExn]
{R(y) = \NULL \mbox{ or } R(y) \notin \DOM(H)}
{\langle H\coma R\coma  \LET x = *y \; \IN s, n\rangle
    \longrightarrow_{D} \MEMEX}
\end{minipage}

\infrule[Sem-FreeExn]
{R(x) = \NULL \mbox{ or } R(x) \notin \DOM(H)}
{\langle H\coma R\coma \Free(x) , n\rangle \xlongrightarrow{\Free}_{D} \MEMEX}

\infrule[Sem-ConstExn]
        { H(R(x)) \neq H'(R'(x))
          \Rtab
          \langle H, R, s, n \rangle \xlongrightarrow{\rho}_{D} \langle H', R', s', n' \rangle}
{\langle H, R, \scon\Sirx s, n \rangle \xlongrightarrow{\rho}_{D} \CONSTEX }

\infax[Sem-OutOfMem]
{ \langle H\coma R\coma \LET x = \Malloc() \ \IN s ,  0  \rangle    \xlongrightarrow{\Malloc}_{D}
  \OVERFLOW}

\end{minipage}

\caption{Operational semantics of \(\mathcal{L}\).}
\label{fig:transitionRules}
\end{figure}
