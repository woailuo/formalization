\section{Introduction}
\label{sec:introduction}
Manual memory mangagement primitives (e.g. \(\texttt{malloc}\) and
\(\texttt{free}\) in C language) are a very flexible way to manage
computer memory cells.  We can write a program which dynamically
allocates a memory cell during running and deallocates a memory cell
when it is no longer used. However, manual memory management
primitives often cause hard-to-find problems, for example, double
frees (\texttt{free} a deallocated memory cell ), memory leaks (forget
to deallocate memory cells) and illegal accesses to a dangling
pointer. Therefore, many static verification methods have been
proposed to guarantee safe memory deallocation. They prove
\emph{partial} memory-leak freedom: if a program terminates, all the
memory cells are safe deallocated. As we know that nonterminating
programs are very common in real-world programmings such as Web
servers and operating systems. To guarantee \emph{total} memory-leak
freedom, if a program does not consume unbounded number of memory
cells during execution, is a very crucial issue.

\begin{exmp}\label{ex:ex1}
\begin{figure}[h]
1  \Rtab $h()$= \dtb \dtb\dtb\Rtab$h'()$= \\
2  \dtb $\LET \; x = \MALLOC  \; \IN$ \dtb \Rtab$\LET \; x = \MALLOC  \; \IN$\\
3  \dtb $\LET \; y = \MALLOC  \; \IN$ \dtb \Rtab$\LET \; y = \MALLOC  \; \IN$\\
4  \dtb $\Free(x)$; $\Free(y) $;\;$h()$ \dtb \Rtab\ \ $h'()$; $\Free\Cirx$; \ $\Free(y)$
\caption{Memory leaks in nonterminating programs.}
\label{ex:np}
\end{figure}
Figure~\ref{ex:np} describes partial and total memory-leak freedom.
Both \(h\) and \(h'\) are partially memory-leak free because they do
not terminate.  The function \(h\) is totally memory-leak free since
it consumes at most two cells\footnote{We assume that every memory
  cell allocated by \(\Malloc\) is fixed size to simplify our type
  system introduced in Section~\ref{sec:typesystem}.  Extension with
  variable-length cells is one of our future work.}.  However, the
function \(h'\), when it is invoked, consumes unbounded number of
memory cells; hence \(h'\) is not totally memory-leak free.
\end{exmp}

In order to prove \emph{total} memory deallocation, we proposed a
behavioral type system in previous study[]. It can abstract the
behavior of a program by using sequential process whose actions
represent manual memory management primitives. For example, the
abstract behavioral type of function \(h\) in Figure~\ref{ex:np} is
\(\mu\alpha.\Malloc;\Malloc;\Free;\Free;\alpha\), which represents
function \(h\) allocates two memory cells, deallocates them, and then
recursively call itself again; the behavioral type of function \(h'\)
is abstracted as \(\mu\alpha.\Malloc;\Malloc;\alpha;\Free;\Free\),
which represents \(h'\) allocates two memory cells, call itself again,
and then deallocates those two cells. From these two examples you may
notice that our behavior type only consider the the number and order
of manual memory management primitives and recursively calls. That way
we can easily estimate the upper bound of memory cells a program
consumed.

Although our previours behavioral type system can abstract the
behavior of a program and estimate the upper bound of memory
consumption, verification on abstracted behavioral types are failed in
some cases. For example, the extracted behavioral type of function
\(foo\) in Figure~\ref{ex:np2} is \( \mu\alpha.\Malloc;\Malloc;\Malloc
+ \TSKIP;\Free + \TSKIP;\Free;\Free;\alpha \), which expresses that
function \(foo\) allocates two memory cells, a choice command between
allocating one memory cell and skipping, a choice command between
deallocating one cell and skipping, deallocates two cells, and then
call itself again. Due to the choice behavioral type, the above type
may be seen as \(
\mu\alpha.\Malloc;\Malloc;\Malloc;\TSKIP;\Free;\Free;\alpha \), which
expresses function \(foo\) consumes three memory cells but deallocates
two memory cells, and then iterates this behavior again. This behavior
means function \(foo\) consumes unbounded number of memory cells,
although the original program is \emph{total} memory-leak freedom.

\begin{exmp}\label{ex:ex2}
\begin{figure}[h]
1  \dtb $foo()$ =  \\
2    \dtb \dtb  $\LET y = \MALLOC \IN $ \\
3    \dtb \dtb  $\LET x = \MALLOC \IN $\\
4    \dtb \dtb  $\IFNULL(*y) \ \THEN \ \SKIP \ \ELSE \LET x_1 = \MALLOC \IN  *x \leftarrow x_1  ;$ \\
5   \dtb \dtb   $\IFNULL(*y) \ \THEN \ \SKIP \ \ELSE \Free(*x) ;$ \\
6   \dtb \dtb   $\Free(x)$ ; $\Free(y)$ ; $foo()$
\caption{a nonterminating program with conditionals}
\label{ex:np2}
\end{figure}
Figure~\ref{ex:np2} describes that functioin \(foo\) is a total
memory-leak freedom program, because it consumes at most three memory
cells during execution. This function has two conditionals: if \(*y\)
is not a null pointer, it will allocates one cell at first conditional
and deallocates that cell at second conditional, otherwise skips.
\end{exmp}

My current idea is to extend previous behavioral type system with
dependent types[]. The dependent types takes more precise information
than traditional types, for example, the type
\((*x)(\Malloc, \TSKIP)\) is a dependent type, because it dependents
on the value \((*x)\). See the function \(foo\) again, the current
behavioral type of it is \(
\mu\alpha.\Malloc;\Malloc;(*x)(\Malloc, \TSKIP);(*x)(\Free, \TSKIP);\Free;\Free;\alpha
\). Therefore, the part\((*x)(\Malloc, \TSKIP);(*x)(\Free, \TSKIP)\)
can be seen as \(\Malloc;\Free\) or \(\TSKIP;\TSKIP\) if \((*x)\) does
not change between these two choices, which expresses total
memory-leak freedom.

The reminder of this paper is structured as
follows. Section~\ref{sec:language} describes an imperative language
with allocation and deallocation primitives and its operational
semantics. Section~\ref{sec:typesystem} introduces the extended
behavioral type system with dependent
types. Section~\ref{sec:reconstruction} describes the type
reconstruction procedure; Section~\ref{sec:experiments} shows some
experiments and give a discussion; Section~\ref{sec:relatedwork}
describes the related works; Section~\ref{sec:conclusion} concludes
this paper.
